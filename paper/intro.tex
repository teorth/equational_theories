\section{Introduction}

\subsection{Magmas and equational laws}

A \emph{magma} $M = (M,\op)$ is a set $M$ (known as the \emph{carrier}) together with a binary operation $\op \colon M \times M \to M$.  An \emph{equational law} for a magma, or \emph{law} for short, is an identity involving $\op$ and some indeterminates, which we will typically denote using the symbols $x,y,z,u,v,w$.  Familiar examples of equational laws include the \emph{commutative law}
\begin{equation}\label{eq43}\tag{E43}
    x \op y = y \op x
\end{equation}
and the \emph{associative law}
\begin{equation}\label{eq4512}\tag{E4512}
    (x \op y) \op z = x \op (y \op z).
\end{equation}
For our project we have assigned a unique number to each equational law, which we describe in \Cref{numbering-app}.  Formally, one can represent an equational law syntactically as a string $w_1 \formaleq w_2$, where $w_1, w_2$ are words in a free magma generated by formal indeterminate symbols; see ???.

A magma $M$ obeys a law $E$ if the law $E$ holds for all possible assignments of the indeterminate to $M$, in which case we write $M \models E$.  Thus for instance $M \models E43$ if one has $x \op y = y \op x$ for all $x,y \in M$.

We place a pre-order on laws by writing $E \leq E'$ or $E \vdash E'$ if every magma that obeys $E$, also implies $E'$: $(M \models E) \implies (M \models E')$.  We say that two laws are \emph{equivalent} if they imply each other.  For instance, the constant law
\begin{equation}\label{eq46}\tag{E46}
x \op y = z \op w
\end{equation}
can easily be see to be equivalent to the law
\begin{equation}\label{eq41}\tag{E41}
x \op x = y \op z.
\end{equation}
In this pre-ordering, a maximal element is given by the trivial law
\begin{equation}\label{eq1}\tag{E1}
x = x
\end{equation}
and a minimal element is given by the singleton law
\begin{equation}\label{eq2}\tag{E2}
x = y.
\end{equation}

The \emph{order} of an equational law is the number of occurrences of the magma operation.  For instance, the commutative law \eqref{eq43} has order $2$, while the associative law \eqref{eq4512} has order $4$.  We note some selected laws of small order that have previously appeared in the literature:
\begin{itemize}
\item The \emph{central groupoid law}
\begin{equation}\label{eq168}\tag{E168}
x = (y \op x) \op (x \op z)
\end{equation}
is an order $3$ law introduced by Evans \cite{evans} and studied further by Knuth \cite{knuth} and many further authors, being closely related to central digraphs (also known as unique path property diagraphs), and leading in particular to the discovery of the Knuth-Bendix algorithm \cite{knuth-bendix}; see \cite{klt} for a more recent survey
\item \emph{Tarski's axiom}
\begin{equation}\label{eq543}\tag{E543}
    x = y \op ( (z \op (x \op (y \op z))))
\end{equation}
is an order $4$ law that was shown by Tarski \cite{Tarski1938} to characterize the operation of subtraction in an abelian group; that is to say, a magma $M$ obeys \eqref{eq543} if and only if there is an abelian group structure on $M$ for which $x \op y = x-y$ for all $x,y \in M$.
\item In a similar vein, it was shown in \cite{mendelsohn-padmanabhan} that the order $4$ law
\begin{equation}\label{eq1571}\tag{E1571}
    x = (y \op z) \op (y \op (x \op z))
\end{equation}
characterizes addition (or subtraction) in an abelian group of exponent $2$; it was shown in \cite{mccune_et_al} that the order $4$ law
\begin{equation}\label{eq345169}\tag{E345169}
    x = (y \op ((x \op y) \op y)) \op (x \op (z \op y))
\end{equation}
characterizes the Sheffer stroke in a boolean algebra, and it was shown in \cite{higman-neumann} that the order $8$ law
\begin{equation}\label{eq42323216}\tag{E42323216}
x = y \op ((((y \op y) \op x) \op z) \op (((y \op y) \op y) \op z))
\end{equation}
characterizes division in a (not necessarily abelian) group.
\end{itemize}

The Birkhoff completeness theorem \cite[Th. 3.5.14]{term-rewriting} implies that an implication $E \vdash E'$ of equational laws holds if and only if the left hand side of $E'$ can be transformed into the right-hand side by a finite number of substitution rewrites using the law $E$.  However, the problem of determining whether such an implication holds is undecidable in general \cite{mckenzie}.Even when the order is small, some implications\footnote{Another contemperaneous example of this phenomenon was the solution of the Robbins problem \cite{robbins}.} can require lengthy computer-assisted proofs; for instance, it was noted in \cite{Kisielewicz2} that the order $4$ law
\begin{equation}\label{eq1689}\tag{E1689}
    x = (y \op x) \op ((x \op z) \op z)
\end{equation}
was equivalent to the singleton law \eqref{eq2}, but all known proofs are computer-assisted.

\subsection{The Equational Theories Project}

As noted in \Cref{numbering-app}, there are $4694$ equational laws of order at most $4$.  In September of 2024, we launched the \emph{Equational Theories Project} (ETP)\footnote{\url{https://teorth.github.io/equational_theories/}} to completely determine the implication pre-ordering $\leq$ for this set of laws.  Ostensibly, this determining the truth or falsity of $4694 \times 4693 = 22028942$ implications; while one can use properties such as the transitivity of the pre-ordering to reduce the work somewhat, this is clearly a task that requires significant automation.

\note{
Describe the initial aims and history of the project. Also discuss extension to finite magma implications (refer to Austin pair section).

Mention BusyBeaver challenge as a precursor

Contrast with "Polymath" projects

Cite Phillips-Vojtechovsky https://arxiv.org/abs/math/0701714 and followup work (for a similar smaller scale project on Moufang loops), as well as this blog post of Wolfram. Also this paper of McCune

Speculative: uncover "Mathematics from Mars", as discussed here.

Summary of equation analysis techniques
Some recapitulation of this thread. Several can be discussed in separate sections afterwards. Perhaps make a separation between easily automatable techniques (that can be used to sweep the entire graph), and techniques that require human attention (which one would only perform either on selected subgraphs, or on survivors of automated sweeps).
}
