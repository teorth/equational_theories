\section{Mathematical Foundations}

\note{This section covers topics like free magmas (including those relative to theories), a completeness theorem, and confluence (unique simplification).}


\section{Formal Foundations}

\note{TODO: expand this sketch.}

    Here we describe the Lean framework used to formalize the project, covering technical issues such as:

\begin{itemize}
    \item Magma operation symbol issues
    \item Syntax (`LawX`) versus semantics (`EquationX`)
    \item "Universe hell" issues
    \item Additional verification (axiom checking, Leanchecker, etc.)
    \item Use of the `conjecture` keyword
    \item Use of namespaces to avoid collisions between contributions. (Note: we messed up with this with FreeMagma! Should have namespaced back end results as well as front end ones.)
    \item Use of Facts command to efficiently handle large numbers of anti-implications at once
\end{itemize}

Upstream contributions:
\begin{itemize}
    \item Mathlib \url{https://github.com/leanprover-community/mathlib4/pulls?q=is%3Apr+is%3Abody+EquationalTheories+}
    \item LeanBlueprint \url{https://github.com/PatrickMassot/leanblueprint/pulls?q=is%3Apr+in%3Abody+EquationalTheories+}
\end{itemize}
