\section{Numbering system}\label{numbering-app}

In this section we record the numbering conventions we use for equational laws.

For this formal definition we use the natural numbers $0,1,2,\dots$ to represent and order indeterminate variables; however, in the main text, we use the symbols $\x,\y,\z,\w,\uu,\vv,\mathrm{r},\mathrm{s},\mathrm{t}$ instead (and do not consider any laws with more than eight variables).

To define the ordering we use on equational laws, we first consider the case where there is a single indeterminate $\ast$.
We place a well-ordering on words $w,w'$ with a single indeterminate $\ast$ by declaring $w > w'$ if one of the following holds:
\begin{itemize}
    \item $w$ has a larger order than $w'$.
    \item $w = w_1 \op w_2$ and $w' = w'_1 \op w'_2$ have the same order $n \geq 1$ with $w_1 > w'_1$.
    \item $w = w_1 \op w_2$ and $w' = w'_1 \op w'_2$ have the same order $n \geq 1$ with $w_1 = w'_1$ and $w_2 > w'_2$.
\end{itemize}
Thus, for instance
$$ \ast < \ast \op \ast < \ast \op (\ast \op \ast) < (\ast \op \ast) \op \ast.$$

We similarly place a well-ordering on equational laws $w_1 \formaleq w_2$ with a single indeterminate $\ast$ by declaring $w_1 \formaleq w_2 > w'_1 \formaleq w'_2$ if one of the following holds:
as follows:
\begin{itemize}
\item  $w_1 \formaleq w_2$ has a longer order than $w'_1 \formaleq w'_2$.
\item If $w_1 \formaleq w_2$ has the same order as $w'_1 \formaleq w'_2$, and $w_1 > w'_1$.
\item If $w_1 \formaleq w_2$ has the same order as $w'_1 \formaleq w'_2$, $w_1 = w'_1$, and $w_2 > w'_2$.
\end{itemize}
Thus for instance
$$ (\ast \op \ast \formaleq \ast \op (\ast \op \ast)) < (\ast \op \ast \formaleq (\ast \op \ast) \op \ast).$$

Finally for equational laws with alphabet $\x,\y,\z,\w,\uu,\vv,\mathrm{r},\mathrm{s},\mathrm{t}$, define the \emph{shape} of that law to be the law formed by replacing all indeterminates with $\ast$; for instance, the shape of \eqref{eq4512}, $\x \op (\y \op \z) = (\x \op \y) \op \z$, is $\ast \op (\ast \op \ast) \formaleq (\ast \op \ast) \op \ast$.  We then place a well-ordering $w_1 \formaleq w_2$ with indeterminates $\x,\y,\z,\w,\uu,\vv,\mathrm{r},\mathrm{s},\mathrm{t}$ by declaring $w_1 \formaleq w_2 > w'_1 \formaleq w'_2$ if one of the following holds:
\begin{itemize}
\item The shape of $w_1 \formaleq w_2$ is greater than the shape of $w'_1 \formaleq w'_2$.
\item $w_1 \formaleq w_2$ and $w'_1 \formaleq w'_2$ have the same shape, and the string of variables appearing in $w_1 \formaleq w_2$ is lower in the lexicographical ordering (using $\x < \y < \z < \w < \uu < \vv < \mathrm{r} < \mathrm{s} < \mathrm{t}$) than the corresponding string for $w'_1 \formaleq w'_2$.
\end{itemize}
Thus for instance any law of shape $\ast \op \ast \formaleq \ast \op (\ast \op \ast)$ is lower than any law of shape
$\ast \op \ast \formaleq (\ast \op \ast) \op \ast$.  Among the laws of shape $\ast \op \ast \formaleq \ast \op (\ast \op \ast)$, the lowest is $\x \op \x \formaleq \x \op (\x \op \x)$, which is less than (say) $\x \op \x \formaleq \y \op (\y \op \y)$, which is in turn less than $\x \op \y \formaleq \x \op (\x \op \x)$.

We say that two equational laws are \emph{definitionally equivalent}\footnote{This can be distinguished from the weaker notion of \emph{propositional equivalence} (mutual entailment) used in the rest of the paper.} if one can be obtained from another by some combination of relabeling the variables and applying the symmetric law $w_1 \formaleq w_2 \iff w_2 \formaleq w_1$.  For instance, $(0 \op 1) \op 2 \formaleq 1$ is definitionally equivalent to $0 \formaleq (1 \op 0) \op 2$.  We then replace every equational law with their minimal element in their definitional equivalence class, which can be viewed as the \emph{normal form} for that law; for instance, the normal form of $(0 \op 1) \op 2 \formaleq 1$ would be $0 \formaleq (1 \op 0) \op 2$.  Finally, we eliminate any law of the form $w \formaleq w$ other than $0 \formaleq 0$.  We then number the remaining equations $\Eq{1}, \Eq{2}, \dots$.  For instance, $\Eq{1}$ is the trivial law $0 \formaleq 0$, $\Eq{2}$ is the constant law $0 \formaleq 1$, $\Eq{3}$ is the idempotent law $0 \formaleq 0 \op 0$, and so forth.  Lists and code for generating these equations, or the equation number attached to a given equation, can be found in the ETP repository.

The number of equations in this list of order $n=0,1,2,\dots$ is given by
$$ 2, 5, 39, 364, 4284, 57882, 888365, \dots$$
(\url{https://oeis.org/A376640}).  The number can be computed to be
$$ C_{n+1} B_{n+2}/2$$
if $n$ is odd, $2$ if $n=0$, and
$$ (C_{n+1} B_{n+2}+ C_{n/2}(2D_{n+2}-B_{n+2}))/2 - C_{n/2} B_{n/2+1}$$
if $n > 2$ is even, where $C_n, B_n$ are the Catalan and Bell numbers, and $D_n$ is the number of partitions of $[n]$ up to reflection, which for $n=0,1,2,\dots$ is
$$ 1, 1, 2, 4, 11, 32, 117, \dots$$
(\url{https://oeis.org/A103293}).  A proof of this claim can be found in the ETP blueprint.  In particular, there are $4694$ equations of order at most $4$.

Below we record some specific equations appearing in this paper, using the alphabet $\x$, $\y$, $\z$, $\w$, $\uu$, $\vv$ in place of $0$, $1$, $2$, $3$, $4$, $5$, $\dots$ for readability.
\begingroup\allowdisplaybreaks
\begin{align}
    \x &\formaleq \x & \hbox{(Trivial law)} \label{eq1}\tag{E1} \\
    \x &\formaleq \y & \hbox{(Singleton law)} \label{eq2}\tag{E2} \\
    \x &\formaleq \x \op \x & \hbox{(Idempotent law)} \label{eq3}\tag{E3} \\
    \x &\formaleq \x \op \y & \hbox{(Left-absorptive law)} \label{eq4}\tag{E4} \\
    \x &\formaleq \y \op \x & \hbox{(Right-absorptive law)} \label{eq5}\tag{E5} \\
    \x &\formaleq \x \op (\y \op \x) \label{eq10}\tag{E10} \\
    \x &\formaleq \x \op (\y \op \y) & \hbox{(Right-unit squares law)} \label{eq11}\tag{E11} \\
    \x &\formaleq (\x \op \x) \op \x \label{eq23}\tag{E23} \\
    \x \op \x &\formaleq \y \op \y & \hbox{(Unipotence law)} \label{eq40}\tag{E40} \\
    \x \op \x &\formaleq \y \op \z & \hbox{(Constant law)} \label{eq41}\tag{E41} \\
    \x \op \y &\formaleq \y \op \x & \hbox{(Commutative law)} \label{eq43}\tag{E43} \\
    \x \op \y &\formaleq \z \op \w & \hbox{(Constant law)} \label{eq46}\tag{E46} \\
    \x &\formaleq \x \op (\x \op (\x \op \x)) \label{eq47}\tag{E47} \\
    \x &\formaleq \y \op (\y \op (\x \op \y))  \label{eq73}\tag{E73} \\
    \x &\formaleq (\x \op \x) \op (\x \op \x) \label{eq151}\tag{E151} \\
    \x &\formaleq (\y \op \x) \op (\x \op \z) & \hbox{(Central groupoid law)} \label{eq168}\tag{E168} \\
    \x &\formaleq (\x \op (\x \op \y)) \op \y \label{eq206}\tag{E206} \\
    \x &\formaleq ((\x \op \x) \op \x) \op \x \label{eq255}\tag{E255} \\
    \x \op \y &\formaleq \x \op (\y \op \z) & \hbox{(Right reduction law)} \label{eq327}\tag{E327} \\
    \x \op \y &\formaleq (\x \op \y) \op \y & \hbox{(Right idempotence law)} \label{eq378}\tag{E378} \\
    \x \op \y &\formaleq (\z \op \x) \op \y & \hbox{(Left reduction law)} \label{eq395}\tag{E395} \\
    \x &\formaleq \x \op (\x \op (\x \op (\y \op \x))) \label{eq413}\tag{E413} \\
    \x &\formaleq \y \op (\z \op (\x \op (\y \op \z))) & \hbox{(Tarski's axiom)} \label{eq543}\tag{E543} \\
    \x &\formaleq \y \op (\x \op ((\y \op \x) \op \y)) & \hbox{(Last open implication)} \label{eq677}\tag{E677} \\
    \x &\formaleq \x \op ((\x \op \x) \op (\x \op \x)) \label{eq817}\tag{E817} \\
    \x &\formaleq \x \op ((\y \op \z) \op (\x \op \z)) \label{eq854}\tag{E854}\\
    \x &\formaleq \x \op ((\y \op (\y \op \x)) \op \x) \label{eq1045}\tag{E1045} \\
    \x &\formaleq \x \op ((\y \op (\z \op \x)) \op \x) \label{eq1055}\tag{E1055} \\
    \x &\formaleq \y \op ((\y \op (\x \op \x)) \op \y) \label{eq1110}\tag{E1110} \\
    \x &\formaleq \y \op ((\y \op (\x \op \z)) \op \z) \label{eq1117}\tag{E1117} \\
    \x &\formaleq \y \op (((\x \op \y) \op \x) \op \y) \label{eq1286}\tag{E1286} \\
    \x &\formaleq (\y \op \x) \op (\x \op (\z \op \y)) & \hbox{(Weak central groupoids)}\label{eq1485}\tag{E1485} \\
    \x &\formaleq (\y \op \z) \op (\y \op (\x \op \z)) & \hbox{(Boolean groups)} \label{eq1571}\tag{E1571} \\
    \x &\formaleq (\x \op \x) \op ((\x \op \x) \op \x) \label{eq1629}\tag{E1629} \\
    \x &\formaleq (\x \op \y) \op ((\x \op \y) \op \y) \label{eq1648}\tag{E1648} \\
    \x &\formaleq (\x \op \y) \op ((\y \op \y) \op \z) \label{eq1659}\tag{E1659} \\
    \x &\formaleq (\y \op \x) \op ((\x \op \z) \op \z) & \hbox{(Equivalent to~\eqref{eq2})} \label{eq1689}\tag{E1689} \\
    \x &\formaleq (\y \op \y) \op ((\y \op \x) \op \y) \label{eq1729}\tag{E1729} \\
    \x &\formaleq (\y \op (\x \op (\y \op \x))) \op \y \label{eq2301}\tag{E2301} \\
        \x &\formaleq (\x \op ((\x \op \x) \op \x)) \op \x \label{eq2441}\tag{E2441} \\
        \x &\formaleq ((\y \op (\x \op \y)) \op \x) \op \y & \hbox{(Dual of \eqref{eq677})} \label{eq2910}\tag{E2910} \\
        \x \op \y &\formaleq \x \op (\y \op (\x \op \y)) \label{eq3316}\tag{E3316} \\
        \x \op \y &\formaleq (\x \op \z) \op (\y \op \z) \label{eq3737}\tag{E3737} \\
        \x \op \y &\formaleq (\x \op (\y \op \x)) \op \y \label{eq3925}\tag{E3925} \\
        \x \op (\y \op \x) &\formaleq \x \op (\y \op \z) \label{eq4315}\tag{E4315} \\
        \x \op (\x \op \x) &\formaleq (\x \op \x) \op \x & \hbox{(Cube-associativity law)} \label{eq4380}\tag{E4380} \\
        \x \op (\y \op \y) &\formaleq (\y \op \y) \op \x & \hbox{(Central squares law)} \label{eq4482}\tag{E4482} \\
        \x \op (\y \op \z) &\formaleq (\x \op \y) \op \z & \hbox{(Associative law)} \label{eq4512}\tag{E4512} \\
        \x \op (\y \op \z) &\formaleq (\y \op \z) \op \x & \hbox{(Central products law)} \label{eq4531}\tag{E4531} \\
        \x &\formaleq \y \op (\y \op (\y \op (\x \op (\z \op \y)))) \label{eq5093}\tag{E5093} \\
        \x &\formaleq \y \op (\z \op ((\y \op \x) \op (\z \op (\y \op \z)))) & \hbox{(Eisenstein modules)} \label{eq85914} \tag{E85914} \\
        \x &\formaleq \y \op (\z \op ((\y \op \z) \op (\w \op (\x \op \w)))) & \hbox{(Gaussian modules)} \label{eq86082} \tag{E86082} \\
        \x &\formaleq (\y \op ((\x \op \y) \op \y)) \op (\x \op (\z \op \y)) & \hbox{(Sheffer stroke)} \label{eq345169}\tag{E345169}
\end{align}
We also list some order-$8$ characterizations of group division relevant for \Cref{higman-neumann}.
\begin{align}
        \x &\formaleq \y \op ((((\x \op \x) \op \x) \op \z) \op (((\x \op \x) \op \y) \op \z)) & \hbox{(McCune law)} \label{eq42302852}\tag{E42302852} \\
        \x &\formaleq \y \op ((((\x \op \x) \op \x) \op \z) \op (((\y \op \y) \op \y) \op \z)) & \label{eq42302946}\tag{E42302946} \\
        \x &\formaleq \y \op ((((\y \op \y) \op \x) \op \z) \op (((\y \op \y) \op \y) \op \z)) & \hbox{(Higman--Neumann law)} \label{eq42323216}\tag{E42323216} \\
        \x &\formaleq (\y \op \y) \op (\y \op ((\x \op \z) \op (((\x \op \x) \op \y) \op \z))) & \hbox{(in finite magmas)} \label{eq67953597}\tag{E67953597} \\
        \x &\formaleq ((\y \op \y) \op \y) \op ((((\x \op \x) \op \x) \op \z) \op (\y \op \z)) & \label{eq89176740}\tag{E89176740} \\
        \x &\formaleq ((\y \op \y) \op ((\x \op \z) \op \x)) \op ((\z \op \w) \op (\x \op \w)) & \label{eq102744082}\tag{E102744082} \\
        \x &\formaleq ((\y \op \y) \op (\y \op (\x \op (((\y \op \y) \op \y) \op \z)))) \op \z & \hbox{(McCune law)} \label{eq147976245}\tag{E147976245}
\end{align}
\endgroup
