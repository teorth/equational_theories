\section{Conclusions and Future Directions}

\note{TODO: Expand this sketch}

Insights learned, future speculation. Utilize the discussion on future directions. Some ideas from that list:

\begin{itemize}
  \item A database of triple implications (EquationX and EquationY imply EquationZ) - see also this discussion.
  \item Are there any equational laws that have no non-trivial finite models, but have surjunctive models?
  \item Can we produce interesting examples of irreducible implications EquationX -> EquationY (no intermediate EquationZ can interpose)
  \item Degree of satisfiability results, e.g., if a central groupoid obeys the natural central groupoid axiom 99\% of the time, is it a natural central groupoid?
  \item Kisielewicz's question: does 5093 have any infinite models?
  \item "Insight mining" the large corpus of automated proofs that have been generated.
  \item How machine-learnable is the implication graph? (See AI/ML section)
\end{itemize}

Note that our choice to focus particularly on some laws and not others is to some extent an artefact of the order in which we discovered and deployed tools.  For instance, by deploying automated theorem provers at an early stage, we might have settled some implications that had more interesting human-readable proofs that we missed.  Similarly, we developed some sophisticated theory for 854 that is now superseded by finite counterexamples; but had the finite counterexamples been discovered first, we would not have found the theoretical arguments.  It may be productive for future work to revisit some portions of the implication graph and locate alternate proofs and methods.


Automation often overtook the rate of human progress, for instance in developing metatheorems. Does this create an opportunity cost? Raised as a possible discussion point here by Zoltan Kocsis.

\section*{Acknowledgments}

Thanks to \href{https://github.com/ClaudMor}{Claudio Moroni} for the exploration of directed link prediction
on the implication graph using GNN autoencoders described in \Cref{sec:dlp}.

We are also grateful to the many additional participants to the Equational Theories Project for their numerous comments and encouragement, including but not restricted to Edward van de Meent, ...
