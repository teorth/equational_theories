\documentclass[12pt]{amsart}

% Packages
\usepackage{amsmath, amssymb, amsthm}
\usepackage{geometry}
\usepackage{hyperref}
\usepackage{cleveref}
\usepackage{graphicx}
\usepackage{enumitem}
\usepackage{color}
\usepackage{mathtools}
\usepackage{tikz}
\usepackage{tikz-cd}
\usepackage{quiver}
\usepackage{mathrsfs}
\usepackage{proof}
\usepackage{todonotes}
\usepackage{siunitx}
\usepackage[utf8]{inputenc}
\usepackage[T1]{fontenc}
\usepackage{fancyvrb}
\fvset{commandchars=\\\{\}}
\usepackage{pmboxdraw}
\usetikzlibrary{positioning,arrows.meta,fit}

\tikzset{
  human/.style   = {->, very thick, color=black, shorten >=2pt},
  auto/.style    = {->, thin, dashed, color=black!60, shorten >=2pt},
  semi/.style    = {->, thick, dash pattern=on 4pt off 2pt on 1pt off 2pt, color=black!80, shorten >=2pt}
}


% Page Setup
\geometry{letterpaper, margin=1in}
\setlength{\parindent}{0pt} % No indent for paragraphs
\setlength{\parskip}{1em}   % Spacing between paragraphs

% Theorem Styles
\newtheorem{theorem}{Theorem}[section]
\newtheorem{lemma}[theorem]{Lemma}
\newtheorem{proposition}[theorem]{Proposition}
\newtheorem{corollary}[theorem]{Corollary}
\theoremstyle{definition}
\newtheorem{definition}[theorem]{Definition}
\newtheorem{example}[theorem]{Example}
\newtheorem{remark}[theorem]{Remark}

% Commands
\newcommand{\R}{\mathbb{R}}
\newcommand{\C}{\mathbb{C}}
\newcommand{\N}{\mathbb{N}}
\newcommand{\Z}{\mathbb{Z}}
\newcommand{\Q}{\mathbb{Q}}
\newcommand{\F}{\mathbb{F}}
\newcommand{\x}{\mathrm{x}}
\newcommand{\y}{\mathrm{y}}
\newcommand{\z}{\mathrm{z}}
\newcommand{\w}{\mathrm{w}}
\newcommand{\uu}{\mathrm{u}}
\newcommand{\vv}{\mathrm{v}}
\newcommand{\op}{\diamond}
\newcommand{\formaleq}{\simeq}
\newcommand{\eps}{\varepsilon}
\newcommand{\vdashfin}{\vdash_{\mathrm{fin}}}
\newcommand{\nvdashfin}{\nvdash_{\mathrm{fin}}}
\newcommand{\note}[1]{{\bf #1}}
\newcommand{\Magma}{{\mathcal{M}}}
\newcommand{\MagmaN}{{\mathcal{N}}}
\newcommand{\Eq}[1]{\mathrm{E}#1}
\newcommand{\TODO}[1]{\todo[inline]{#1}}


% Title Information
\title[Equational Theories Project]{The Equational Theories Project: Advancing Collaborative Mathematical Research at Scale}
\author[Equational Theories Project contributors]{Matthew Bolan, Joachim Breitner, Jose Brox, Mario Carneiro,
  Martin Dvorak, Andr\'es Goens, Aaron Hill, Harald Husum, Zoltan Kocsis, Bruno Le Floch, Lorenzo Luccioli, Douglas McNeil,
  Alex Meiburg, Pietro Monticone, Giovanni Paolini, Marco Petracci, Bernhard Reinke, David Renshaw, Marcus Rossel, Cody Roux,
  J\'er\'emy Scanvic, Shreyas Srinivas, Anand Rao Tadipatri, Terence Tao, Vlad Tsyrklevich,
  Daniel Weber, Fan Zheng}
\date{\today}

\begin{document}

\begin{abstract}
  We report on the \emph{Equational Theories Project} (ETP), an online collaborative pilot project
  to explore new ways to collaborate in mathematics with machine assistance. The project successfully determined all $\num{22028942}$ edges of the implication graph between the $4694$ simplest equational laws on magmas, by a combination of
  human-generated and automated proofs, all validated by the formal proof assistant language
  \emph{Lean}. As a result of this project, several new constructions of magmas obeying specific laws were discovered, and several auxiliary questions were also addressed, such as the effect of restricting attention to finite magmas.
\end{abstract}

\maketitle

\tableofcontents

\section{Introduction}

The purpose of this paper is to report on the \emph{Equational Theories Project} (ETP)\footnote{\url{https://teorth.github.io/equational_theories/}}, a pilot project launched\footnote{\url{https://terrytao.wordpress.com/2024/09/25}} in September 2024 to explore new ways to collaboratively work on mathematical research projects using machine assistance. The project goal, in the area of universal algebra, was selected\footnote{The specific mathematical goal was inspired by \href{https://mathoverflow.net/questions/450930}{a MathOverflow question}.} to be particularly amenable to crowdsourced and computer-assisted techniques, while still being of mathematical research interest. \note{Describe outcomes}

\subsection{Magmas and Equational Laws}

In order to describe the mathematical goals of the ETP, we need some notation. A \emph{magma} $\Magma = (M,\op)$ is a set $M$ (known as the \emph{carrier}) together with a binary operation $\op \colon M \times M \to M$. An \emph{equational law} for a magma, or \emph{law} for short, is an identity involving $\op$ and some formal indeterminates, which we will typically denote using the Roman letters $\x,\y,\z,\w,\uu,\vv$, as well as the formal equality symbol $\formaleq$ in place of the equality symbol $=$ to emphasize the formal nature of the law.

In the ETP, a unique number was assigned to each equational law, via a numbering system that we describe in \Cref{numbering-app}.  For instance, the \emph{commutative law} $\x \op \y \formaleq \y \op \x$ is assigned the equation number \eqref{eq43}, while the \emph{associative law} $(\x \op \y) \op \z \formaleq \x \op (\y \op \z)$ is assigned the equation number \eqref{eq4512}.  A list of all equations referred to by number in this paper is provided in \Cref{numbering-app}.

A magma $\Magma = (M,\op)$ obeys a law $E$ if the law $E$ holds for all possible assignments of the indeterminate to elements of $M$, in which case we write $\Magma \models E$. Thus for instance $\Magma \models E43$ if one has $x \op y = y \op x$ for all $x,y \in M$.  Note that the formal indeterminate symbols $\x, \y$ in $E43$ are now replaced by concrete elements $x,y$ of the carrier $M$.

We say that a law $E$ \emph{entails} or \emph{implies} another law $E'$ if every magma that obeys $E$, also implies $E'$: $(\Magma \models E) \implies (\Magma \models E')$.  We write this relation as $E \vdash E'$. We say that two laws are \emph{equivalent} if they entail each other. For instance, the constant law $\x \op \y \formaleq \z \op \w$ \eqref{eq46} can easily be seen to be equivalent to the law $\x \op \x \formaleq \y \op \z$ \eqref{eq41}.  It is easy to see that $\vdash$ is a pre-order, that is to say a partial order after one quotients by equivalence.

In this entailment pre-ordering, the maximal element is given by the trivial law $\x\formaleq\x$ \eqref{eq1}, and the minimal element is given by the singleton law $\x\formaleq \y$ \eqref{eq2}, thus $E2 \vdash E \vdash E1$ for all laws $E$.

We also define a variant: we say that $E$ \emph{entails} $E'$ \emph{for finite magmas}, and write $E \vdashfin E'$, if every \emph{finite} magma $M$ that obeys $E$, also obeys $E'$.  Clearly, the relation $E \vdash E'$ implies $E \vdashfin E'$; but, as observed by Austin \cite{austin_finite}, the converse is not true in general. :

The \emph{order} of an equational law is the number of occurrences of the magma operation. For instance, the commutative law \eqref{eq43} has order $2$, while the associative law \eqref{eq4512} has order $4$. We note some selected laws of small order that have previously appeared in the literature:
\begin{itemize}
\item The \emph{central groupoid law} $\x \formaleq (\y \op \x) \op (\x \op \z)$ \eqref{eq168} is an order-$3$ law introduced by Evans \cite{evans} and studied further by Knuth \cite{knuth} and many further authors, being closely related to central digraphs (also known as unique path property diagraphs), and leading in particular to the discovery of the Knuth-Bendix algorithm \cite{knuth-bendix}; see \cite{klt} for a more recent survey.
\item \emph{Tarski's axiom} $\x \formaleq \y \op ( (\z \op (\x \op (\y \op \z))))$ \eqref{eq543} is an order-$4$ law that was shown by Tarski \cite{Tarski1938} to characterize the operation of subtraction in an abelian group; that is to say, a magma $\Magma = (M,\op)$ obeys \eqref{eq543} if and only if there is an abelian group structure on $\Magma$ for which $x \op y = x-y$ for all $x,y \in M$.
\item In a similar vein, it was shown in \cite{mendelsohn-padmanabhan} (see also \cite{meredith-prior}) that the order-$4$ law
$\x \formaleq (\y \op \z) \op (\y \op (\x \op \z))$ \eqref{eq1571} characterizes addition (or subtraction) in an abelian group of exponent $2$; it was shown in \cite{mccune_et_al} that the order-$4$ law $\x \formaleq (\y \op ((\x \op \y) \op \y)) \op (\x \op (\z \op \y))$ \eqref{eq345169} characterizes the Sheffer stroke in a boolean algebra, and it was shown in \cite{higman-neumann} that the order-$8$ law
$\x \formaleq \y \op ((((\y \op \y) \op \x) \op \z) \op (((\y \op \y) \op \y) \op \z))$ \eqref{eq42323216} characterizes division in a (not necessarily abelian) group.
\end{itemize}
Some further examples of laws characterizing well-known algebraic structures are listed in \cite{mccune-survey}.

The Birkhoff completeness theorem \cite[Th. 3.5.14]{term-rewriting} implies that an implication $E \vdash E'$ of equational laws holds if and only if the left-hand side of $E'$ can be transformed into the right-hand side by a finite number of substitution rewrites using the law $E$. However, the problem of determining whether such an implication holds is undecidable in general \cite{mckenzie}. Even when the order is small, some implications\footnote{Another contemporaneous example of this phenomenon was the solution of the Robbins problem \cite{robbins}.} can require lengthy computer-assisted proofs; for instance, it was noted in \cite{Kisielewicz2} that the order-$4$ law $\x \formaleq (\y \op \x) \op ((\x \op \z) \op \z)$ \eqref{eq1689} was equivalent to the singleton law \eqref{eq2}, but all known proofs are computer-assisted.  Furthermore, for the finite magma implication relation $E \vdashfin E'$, no analogue of the Birkhoff completeness theorem is available.

\subsection{The Equational Theories Project}

As noted in \Cref{numbering-app}, there are $4694$ equational laws of order at most $4$. The primary mathematical goal of the ETP was to completely determine the \emph{implication graph} for these laws, in which there is a directed edge from $E$ to $E'$ precisely when $E \vdash E'$. As the project progressed, an additional goal was added to determine the slightly larger \emph{finite implication graph}, in which there is a directed edge from $E$ to $E'$ precisely when $E \vdashfin E'$.

Such systematic determinations of implication graphs have been seen previously in the literature; for instance, in \cite{phillips-vojtechovsky}, the relations between $60$ identities of Bol--Moufang type were established, and in the blog post \cite[\S 17]{Wolfram_2022}, some initial steps towards generating this graph for the first hundred or so laws on our list were performed. However, to our knowledge, the ETP is the first project to study such implications at the scale of thousands of laws.

The ETP requires the determination of the truth or falsity of $4694^2 = 22033636$ implications (for both arbitrary magmas and finite magmas); while one can use properties such as the transitivity of entailment to reduce the work somewhat, this is clearly a task that requires significant automation. It was also a project highly amenable to crowdsourcing, in which different participants could work on developing different techniques, each of which could be used to fill out a different part of the implication graph. In this respect, the project could be compared with a Polymath project \cite{Gowers2009}, which used online forums such as blogs and wikis to openly collaborate on a mathematical research problem. However, the Polymath model required human moderators to review and integrate the contributions of the participants, which clearly would not scale to the ETP which required the verification of over twenty million mathematical statements. Instead, the ETP was centered around a Github repository in which the formal mathematical contributions had to be entered in the proof assistant language \emph{Lean}, where they could be automatically verified. In this respect, the ETP was more similar to the recently concluded Busy Beaver Challenge\footnote{\url{https://bbchallenge.org/}}, which was a similarly crowdsourced project that computed the fifth Busy Beaver number $BB(5)$ to be $47176870$ through an analysis of about $180$ million Turing machines, with the halting analysis being verified in a variety of computer languages, with the final formal proof written in the proof assistant language \emph{Coq}. One of the aims of the ETP was to explore potential workflows for such collaborative, formally verified mathematical research projects that could serve as a model for future projects of this nature.

Secondary aims of the ETP included the possibility of discovering unusually interesting equational laws, or new experimental observations about such laws, that had not previously been noticed in the literature; and to develop benchmarks to assess the performance of automated theorem provers and other AI tools.

\subsection{Outcomes}

\note{This text assumes (optimistically) that both the original and finite implication graph will be completely formalized.}

The ETP achieved its primary objectives, with all of the implications for both arbitrary magmas and finite magmas formalized in the proof assistant language \emph{Lean}, and can be found on the ETP GitHub repository.  See \Cref{fig:854}, \Cref{fig:1729} for some small fragments of the implication graphs produced. The experience of running such a large collaborative research project introduced several challenges, which we report upon in \Cref{project-sec}. Also, a variety of methods with varying degrees of automation or computer-assistance had to be developed to resolve all the implications, which had quite a variety of difficulty levels.

\begin{figure}
\centering
\includegraphics[width=0.85\textwidth]{854.png}
\caption{A Hasse diagram of all the equational laws implied by \eqref{eq854} (for unrestricted magmas).  An edge in this diagram indicates that the lower equation implies the higher one. Rounded rectangles indicate groups of equivalent laws.  This graph was produced by the visualization tool \emph{Graphiti}, which was developed for this project.}
\label{fig:854}
\end{figure}

\begin{figure}
    \centering
    \includegraphics[width=0.4\textwidth]{ramanujan-infinite.png}
    \includegraphics[width=0.4\textwidth]{ramanujan-finite.png}
    \caption{A Hasse diagram of all the equational laws implied by \eqref{eq1729}, both for unrestricted magmas (left) and finite magmas (right). Note the slightly larger number of implications in the latter.}
    \label{fig:1729}
\end{figure}


Of the $22033636$ possible implications $E \vdash E'$, $8178279$ (or $37.12\%$) would end up being true; for a slightly larger set  {\bf give statistics}, the weaker implication $E \vdashfin E'$ held. To establish such positive implications $E \vdash E'$ or $E \vdashfin E'$, the main techniques used were as follows:

\begin{itemize}
    \item A very small number of positive implications were established and formalized by hand, mostly through direct rewriting of the laws; but this approach would not scale to the full project.
    \item Simple rewriting rules, for instance based on the observation that any law of the form $\x \formaleq f(\y,\z,\dots)$ was necessarily equivalent to the trivial law \eqref{eq2}, could already reduce the size of potential equivalence classes by a significant fraction. We discuss this method in \Cref{rewrite-sec}.
    \item The preorder axioms for $\vdash$, as well as the ``duality'' symmetry of the preorder with respect to replacing a magma operation $x \op y$ with its opposite $x \op^{\mathrm{op}} y \coloneqq y \op x$, can be used to significantly cut down on the number of implications that need to be proven explicitly; ultimately, only $10657$ ($0.05\%$) of the positive implications needed a direct proof. \note{Update these stats when we obtain our final theorem}
    \item To obtain additional implications for finite magmas, heavy reliance was made on the fact that for functions $f \colon M \to M$ on a finite set $M$, surjectivity was equivalent to injectivity.  Some more sophisticated variants of this idea can lead to additional implications; see \Cref{finite-sec}.
    \item Automated Theorem Provers (ATP) could be deployed at extremely fast speeds to establish a complete generating set of positive implications; see \Cref{automated-sec}.
\end{itemize}

More challenging were the $13855357$ ($62.88\%$) implications that were false, $E \not \vdash E'$, and particularly the slightly smaller set of {\bf give stats here} implications that were false even for finite magmas, $E \not \vdashfin E'$. Here, the range of techniques needed to refute such implications were quite varied.
\begin{itemize}
        \item Syntactic methods, such as observing an ``matching invariant'' of the law $E$ that was not shared by the law $E'$, could be used to obtain some refutations.  For instance, if both sides of $E$ had the same order, but both sides of $E'$ did not, this could be used to syntactically refute $E \vdash E'$.  Similarly, if the law $E$ was confluent, enjoyed a complete rewriting system, or otherwise permitted some understanding of the free magma associated to that law, one could decide the assertions $E \vdash E'$ for all possible laws $E'$, or at least a significant fraction of such laws.  We discuss these methods, and the extent to which they can be automated in \Cref{syntactic-sec}.
        \item Small finite magmas, which can be described explicitly by multiplication tables, could be tested by brute force computations to provide a large number of finite counterexamples to implications, or by ATP-assisted methods. See \Cref{finite-sec}.
        \item Linear models, in which the magma operation took the form $x \op y = ax + by$ for some (commuting or non-commuting) coefficients $a,b$, allowed for another large class of counterexamples to implications, which could be automatically scanned for either by brute force or by Grobner basis type calculations; many of these examples could also be made finite. See \Cref{linear-sec}.
        \item Translation invariant models, in which the magma operation took the form $x \op y = x + f(y-x)$ on an additive group, or $x \op y = x f(x^{-1} y)$ on a non-commutative group, reduce matters to analyzing certain functional equations; see \Cref{translation-sec}.
        \item Greedy methods, in which either the multiplication table $(x,y) \mapsto x \op y$ or the function $f$ determining a translation-invariant model are iteratively constructed by a greedy algorithm subject to a well-chosen ruleset, were effective in resolving many implications not easily disposed of by preceding methods. See \Cref{greedy-sec}.
        \item Starting with a simple base magma $\Magma$ obeying both $E$ and $E'$, and either enlarging it to a larger magma $\Magma'$ containing $\Magma$ as a submagma, extending it to a magma $\MagmaN$ with a projection homomorphism $\pi: \MagmaN \to \Magma$, or modifying the multiplication table on a small number of values, also proved effective when combined with greedy methods or with a ``magma cohomology'' construction. See \Cref{modify-base}.
        \item To each equation $E$ one can associate a ``twisting semigroup'' $S_E$.  If $S_E$ is larger than $S_{E'}$, then this can often be used to disprove the implication $E \vdash E'$; see \Cref{twisting-sec}.
        \item Some \emph{ad hoc} models based on existing mathematical objects, such as infinite trees, rings of polynomials, or ``Kisielewicz models'' utilizing the prime factorization of the natural numbers, could also handle some otherwise difficult cases.  In some cases, the magma law induced some relevant and familiar structures, such as a directed graph or a partial order, which also helped guide counterexample constructions. We will not detail these diverse examples here, but refer the reader to the ETP blueprint for more discussion.
        \item Automated theorem provers were helpful in identifying which simplifying axioms could be added to the magma without jeopardizing the ability to refute the desired implication $E \vdash E'$ or $E \vdashfin E'$.
\end{itemize}

In the course of completing the implication graph, some interesting new algebraic structures were discovered.  One such example concerns the magmas obeying \eqref{eq1485}, which we refer to as \emph{weak central groupoids} as they contain the central groupoids (obeying \eqref{eq168}) as a subclass.  In \cite{knuth} it was observed that all finite central groupoids have order equal to a perfect square $n^2$; empirically, we have found that finite weak central groupoids always have order $n^2$ or $2n^2$, although we have no rigorous proof of this claim; they also have a graph-theoretic interpretation analogous to the interpretation of central groupoids as digraphs with the unique path property.  For these and other observations we refer the reader to \href{https://teorth.github.io/equational_theories/blueprint/weak-central-groupoids-chapter.html}{the blueprint of the ETP}.

The objective of using the data from the ETP to establish well-calibrated benchmarks to evaluate ATPs remains an interesting open problem; the participants of this project did not have the required expertise to develop and test such benchmarks to the standards expected in the area.  However, in \Cref{automated-sec} we present a more informal ``field report'' of our experiences using ATPs in the project, in the hope that this will provide some useful guidance to other researchers seeking to incorporate ATPs into their own research.


\subsection{Extensions}

While the primary objective of the ETP was being completed, some additional related results were generated as spinoffs.  Specifically:
\begin{itemize}
\item In \Cref{order-5} we report on classifying which of the $57882$ distinct laws of order $5$ are equivalent to the singleton law \eqref{eq2}, either with or without the requirement that the magma be finite.
\item In \Cref{higman-neumann} we report on classifying the laws of order $8$ that are equivalent to the Higman-Neumann law \eqref{eq42323216}.
\end{itemize}

\note{Also mention ML stuff, GUI}

\section{Notation and Mathematical Foundations}

If $M$ is a magma, we define the left and right multiplication operators $L_a, R_a \colon M \to M$ for $a \in M$ by the formula
\begin{equation}\label{left-right}
    L_x y = R_y x \coloneqq x \op y.
\end{equation}
We also define the squaring operator $S \colon M \to M$ by
\begin{equation}\label{square-def}
    Sx \coloneqq x \op x = L_x x = R_x x
\end{equation}
and the (right) cubing operator $C \colon M \to M$ by
\begin{equation}\label{cube-def}
    Cx \coloneqq Sx \op x = R_x^2 x.
\end{equation}

If $X$ is an alphabet, we let $M_X$ denote the free magma generated by $X$, thus an element of $M_X$ is either a letter in $X$, or of the form $w_1 \op w_2$ with $w_1,w_2 \in M_X$.  Every function $f \colon X \to M$ into a magma $M$ extends to a unique homomorphism $\varphi_f \colon M_X \to M$.  Formally, an equational law with some indeterminates in $X$ can be written as $w_1 \formaleq w_2$ for some $w_1, w_2 \in X$; a magma $M$ then obeys this law if and only if $\varphi_f(w_1) = \varphi_f(w_2)$ for all $f \colon X \to M$.

Every magma $M$ has an opposite $M^{\mathrm{op}}$, which has the same carrier but the opposite operation $x \op^{\mathrm{op}} y \coloneqq y \op x$.  A magma $M$ obeys an equational law $E$ if and only if its opposite $M^{\mathrm{op}}$ obeys the dual law $E^*$, defined by reversing the all operations.  For instance, the dual of Equation 327,
\begin{equation}\label{eq327}\tag{E327}
    x \op y = x \op (y \op z),
\end{equation}
is the the law $y \op x = (z \op y) \op x$, which in our numbering system we rewrite as Equation 395,
\begin{equation}\label{eq395}\tag{E395}
    x \op y = (z \op x) \op y.
\end{equation}
We then see that the implication graph has a duality symmetry: given two equational laws $E_1,E_2$, we have $E_1 \vdash E_2$ if and only if $E_1^* \vdash E_2^*$.

\section{Formal Foundations}

\note{TODO: expand this sketch.}

Here we describe the Lean framework used to formalize the project, covering technical issues such as:

\begin{itemize}
    \item Magma operation symbol issues
    \item Syntax (`LawX`) versus semantics (`EquationX`)
    \item "Universe hell" issues
    \item Additional verification (axiom checking, Leanchecker, etc.)
    \item Use of the `conjecture` keyword
    \item Use of namespaces to avoid collisions between contributions. (Note: we messed up with this with FreeMagma! Should have namespaced back end results as well as front end ones.)
    \item Use of Facts command to efficiently handle large numbers of anti-implications at once
\end{itemize}

Upstream contributions:

\begin{itemize}
    \item \href{https://github.com/leanprover-community/mathlib4/pulls?q=is%3Apr+is%3Abody+EquationalTheories+}{Mathlib contributions}
    \item \href{https://github.com/PatrickMassot/leanblueprint/pulls?q=is%3Apr+in%3Abody+EquationalTheories+}{LeanBlueprint contributions}
\end{itemize}

\section{Project Management}\label{project-sec}
\newcommand{\TODO}[1]{\todo[inline]{#1}}

\TODO{Shreyas Srinivas and Pietro Monticone have volunteered to take the lead on this section.}

This project is, among other things, an experiment on how to organise large scale collaborations for mathematical work. In this section, we describe several aspects of the mechanics of the collaborative effort.

\subsection{Problems of scale in mathematical collaboration}
In order to understand the difficulties of scaling that arise in large scale collaborations, it helps to revisit how traditional mathematical collaborations work and understand why they may not scale. Every collaboration is unique, and we cannot imagine that any universal template exists. However, some general patterns can be observed in any mathematical collaboration. Usually, a small number of contributors, usually under ten, who may know each other, join forces to tackle some class of problems. Typically the collaborators are academics who share substantial amounts of common knowledge. They discuss the problem at hand together, typically with some shared written medium such as a whiteboard. After several rounds of discussion and refinement, different members of the collaboration come up with different pieces of the solution and discuss how they fit together into a whole. Along the way, each collaborator writes out their specific contributions and reviews that of others. After several iterations of this process they synthesize the results into a single paper. At this point, the collaborators are reasonably confident about the correctness of their work, including theorem statements and proofs, and submit the paper for peer review. Of course, there are many variations to this general template, but the basic cycle of discuss, solve, write, cross-check, and revise process is always present in some form or another.

\TODO{Discuss how the above breaks down when a large and diverse collection of collaborators work over the internet.
    Consider citing the \href{https://euromathsoc.org/code-of-practice}{EMS webpage for code of practice} for joint responsibility rules
}
Discuss topics such as:
\begin{itemize}
    \item Project generation from \href{https://github.com/pitmonticone/LeanProject}{template}
    \item GitHub issue management with \href{https://github.com/teorth/equational_theories/labels}{labels} and \href{https://github.com/users/teorth/projects/1}{task management dashboard}
    \item Continuous integration (builds, blueprint compilation, task status transition)
    \item Pre-push git hooks
    \item Use of blueprint (small note, see \#406: blueprint chapters should be given names for stable URLs)
    \item Use of Lean Zulip (e.g. use of polls)
\end{itemize}

Maybe give some usage statistics, e.g. drawing from \url{https://github.com/teorth/equational_theories/actions/metrics/usage}

Mention that FLT is also using a similar workflow.

\subsection{Handling Scaling Issues}

Also mention some early human-managed efforts ("outstanding tasks", manually generated Hasse diagram, etc.) which suffices for the first one or two days of the project but rapidly became unable to handle the scale of the project.

Mention that some forethought in setting up a GitHub organizational structure with explicit admin roles etc. may have had some advantages if we had done so in the planning stages of the project, but it was workable without this structure (the main issue is that a single person -- Terry -- had to be the one to perform various technical admin actions).

Use of transitive reduction etc.\ to keep the Lean codebase manageable. Note that the project is large enough that one cannot simply accept arbitrary amounts of Lean code into the codebase, as this could make compilation times explode. Also note somewhere that transitive completion can be viewed as directed graph completion on a doubled graph consisting of laws and their formal negations.

Technical debt issues, e.g., complications stemming from an early decision to make Equations.lean and AllEquations.lean the ground truth of equations for other analysis and visualization tools, leading to the need to refactor when AllEquations.lean had to be split up for performance reasons.

Note that the "blueprint" that is now standard for guiding proof formalization projects is a bit too slow to keep up with this sort of project that is oriented instead about proving new results. Often new results are stated and formalized without passing through the blueprint, which is then either updated after the fact, or not at all. So the blueprint is more of an optional auxiliary guiding tool than an essential component of the workflow.

\subsection{Other Design Considerations}

Explain what "trusting" Lean really means in a large project. Highlight the kind of human issues that can interfere with this and how use of tools like external checkers and PR reviews by people maintaining the projects still matters. Provide guidelines on good practices (such as branch protection or watching out for spurious modifications in PRs, for example to the CI). Highlight the importance of following a proper process for discussing and accepting new tasks, avoiding overlaps etc. These issues are less likely to arise in projects with one clearly defined decision maker as in this case, and more likely to arise when the decision making has to be delegated to many maintainers.

Note that despite the guarantees provided by Lean, non-Lean components still contained bugs. For instance, an off-by-one error in an ATP run created a large number of spurious conjectures, and some early implementations of duality reductions (external to Lean) were similarly buggy. "Unit tests", e.g., checking conjectured outputs against Lean-validated outputs, or by theoretical results such as duality symmetry, were helpful, and the Equation Explorer visualization tool also helped human collaborators detect bugs.

Meta: documenting thoughts for the future record is quite valuable. It would be extremely tedious to try to reconstruct real-time impressions long after the fact just from the GitHub commit history and Zulip chat archive.

\subsection{Maintenance}

Describe the role of maintainers and explain why they need to be conversant in the mathematics being formalised, as well as Lean. As such, the role of maintainers is often akin to postdocs or assistant profs in a research group who do some research of their own, but spend much of their time to guide others in performing their tasks, the key difference being that contributors are volunteers who choose their own tasks. Explain the tasks maintainers must perform. Examples:

\begin{itemize}
    \item Reviewing proofs,
    \item Helping with proofs and theorem statements when people get stuck,
    \item Offering suggestions and guidance on how to produce shorter or more elegant proofs,
    \item Ensuring some basic standards are met in proof blocks that make proofs robust to upstream changes,
    \item Creating and maintaining CI processes,
    \item Responding to task requests,
    \item Evaluating theorem and definition formulations (for example unifying many theorem statements into one using FactsSyntax),
    \item Suggesting better ones where possible,
    \item Ensuring that there is no excessive and pointless overlap of content in different contributions \TODO{elaborate on what level of overlap was permissible and what we consider excessive}.
\end{itemize}

\section{Counterexample constructions}

\note{TODO: Expand this sketch}

Describe various sources of example magmas used in counterexamples, including the ones listed here. Note that linear magmas let one assign an "affine scheme" to each law that can be used to rule out many, but not all, implications.

Discuss semi-automated creation of finite counterexamples (as discussed here)

Also note some "negative results" - classes of finite magmas that did not yield many additional refutations, e.g. commutative 5x5 magmas.

Translation-invariant magmas (see e.g., this thread for a nice example). Note: any magma with a transitive symmetry will lift to a translation-invariant model, so this helps explain why these are common examples. Also symmetric models could be slightly more likely to obey various laws than general models due to degree of freedom considerations.

Using SAT solvers to find medium sized finite magmas obeying a given law? See this discussion.

Tree based constructions, see here.

Discuss computational and memory efficiencies needed to brute force over extremely large sets of magmas. SAT solving may be a better approach past a certain size!


\section{Syntactic arguments}\label{syntactic-sec}

Many proofs or refutations of implications (or equivalences) between two equational laws $E,E'$ can be obtained from the syntactic form of the equation.  We discuss some techniques here that were useful in the ETP.

\subsection{Simple rewrites}\label{rewrite-sec}

Many equational laws $E'$ can be formally deduced from a given law $E$ by applying the Lean `rw' tactic to rewrite $E'$ repeatedly by some forward or backward application of $E$ applied to arguments that match some portion of $E$.  For instance, the commutative law \eqref{eq43} clearly implies $x \op (y \op z) = (y \op z) \op x$ \eqref{eq4531}
by a single such rewrite.  A brute force application of such rewrite methods is already able to directly generate about $15,000$ such implications, including many equivalences to the singleton law \eqref{eq2} and the constant law \eqref{eq46}.  After applying transitive closure, this generates about four million further such implications.

A simple observation that already generates many equivalences is that any equation of the form $x = f(y,z,\dots)$ necessarily is equivalent to the trivial law $x = y$; similarly, an equation of the form $f(x,y) = g(z,w,\dots)$ implies $f(x,y) = f(x',y')$; and so forth.

\subsection{Matching invariants}

Fix an alphabet $X$. An \emph{matching invariant} is an assignment $I \colon M_X \to {\mathcal I}$ of an object $I(w) \in {\mathcal I}$ in some space ${\mathcal I}$ to each word $w \in M_X$ with the property that if an equational law $w_1 \formaleq w_2$ has matching invariants $I(w_1)=I(w_2)$, then the same matching $I(w'_1) = I(w'_2)$ holds for any consequence $w'_1 \formaleq w'_2$.  In particular, if one law $I(w_1)=I(w_2)$ and $I(w'_1) \neq I(w'_2)$, then the law $w_1 \formaleq w_2$ does not imply the law $w'_1 \formaleq w'_2$.

A simple example of a matching invariant is the order of a word: if $w_1,w_2$ have the same order, then any rewriting of a word $w$ using the law $w_1 \formaleq w_2$ will preserve its order.  Hence, if $w'_1, w'_2$ are of different orders, then $w_1 \formaleq w_2$ cannot imply $w'_1 \formaleq w'_2$.  For instance, the commutative law \eqref{eq43} cannot imply the left-absorptive law \eqref{eq4}.

One source of matching invariants comes from the free magma $M_X$ of a theory:

\begin{proposition}[Free magmas and matching invariants]\label{free-inv}  Let $\Gamma$ be a theory, and let $\iota_{X,\Gamma} \colon X \to M_{X,\Gamma}$ be the map associated to the free magma $M_{X,\Gamma}$ for that theory.  Then the map $I \colon M_X \to M_{X,\Gamma}$ defined by $I(w) \coloneqq \varphi_{\iota_{X,\Gamma}}(w)$ is an invariant.
\end{proposition}

\begin{proof}  Suppose that $w_1 \formaleq w_2$ entails $w'_1 \formaleq w'_2$, and that $I(w_1) = I(w_2)$.  For any $f \colon X \to M_{X,\Gamma}$, the two maps $\varphi_f, \varphi_{f,\Gamma} \circ \varphi_{\iota_{X,\Gamma}} \colon M_X \to M_{X,\Gamma}$ are both homomorphisms that extend $f$, hence agree by the universal property of $M_X$, as displayed by the following commutative diagram:
\[\begin{tikzcd}
	&& X \\
	\\
	{M_X} && {M_{X,\Gamma}} && {M_{X,\Gamma}}
	\arrow[hook, from=1-3, to=3-1]
	\arrow["{\iota_{X,\Gamma}}"', from=1-3, to=3-3]
	\arrow["f", from=1-3, to=3-5]
	\arrow["{I = \varphi_{\iota_{X,\Gamma}}}", from=3-1, to=3-3]
	\arrow["{\varphi_f}"', curve={height=18pt}, from=3-1, to=3-5]
	\arrow["{\varphi_{f,\Gamma}}", from=3-3, to=3-5]
\end{tikzcd}\]
In particular, the hypothesis $I(w_1)=I(w_2)$ implies that $\varphi_f(w_1) = \varphi_f(w_2)$ for all $f \colon X \to M_{X,\Gamma}$; that is to say, the magma $M_{X,\Gamma}$ obeys the law $w_1 \formaleq w_2$, and hence also $w'_1 \formaleq w'_2$ by hypothesis.  In particular, $\varphi_{\iota_{X,\Gamma}}(w'_1) = \varphi_{\iota_{X,\Gamma}}(w'_2)$, which gives $I(w'_1) = I(w'_2)$ as required.
\end{proof}

\begin{example}  If we take $\Gamma = \{E4\}$ to be the theory of the left-absorptive law \eqref{eq4} as described in \Cref{left-absorb}, then the matching invariant $I(w)$ produced by \Cref{free-inv} is the left-most letter of the alphabet $X$ appearing in the word; for instance $I((x \op y) \op z) = x$.  Thus, for example, the left-absorptive law \eqref{eq4} cannot imply the right-absorptive law $x = y \op x$ \eqref{eq5}
\end{example}

\begin{example}  Let $n \geq 1$ be a positive integer, and consider the theory $\Gamma = \{E43, E4512, E_n\}$ consisting of the commutative law \eqref{eq43}, the associative law \eqref{eq4512}, together with the order $n$ law $L_x^n x = x$.  One can check that the free magma $M_{X,\Gamma}$ can be described as the free group of exponent $n$ with generators $e_x, x \in X$, with associated map $\iota_{X,\Gamma} \colon x \mapsto e_x$.  The associated matching invariant $I(w)$ essentially counts the number of times each letter $x \in X$ appears in the word $w$, modulo $n$; for instance, $I(x \op (y \op x)) = 2e_x + e_y$.  For example, the cubic idempotent law $x = (x \op x) \op x$ \eqref{eq23}
has matching invariants $e_x = 3e_x$ in the $n=2$ case, and hence does not imply the idempotent law $x = x \op x$ \eqref{E3} since $e_x \neq 2e_x$ in the $n=2$ case.
\end{example}

\note{Give some statistics on how many refutations can be established by these methods.}

\subsection{Confluence}

\note{Define a confluent law and give some examples.}

\subsection{Complete rewriting systems}

\note{Define a complete rewriting system and give some examples.}

\subsection{Unique factorization}

Discuss the 854 example

\section{Proof automation}\label{automated-sec}

In this project we used proof automation in two ways: automated theorem provers (ATPs) and \emph{Lean} tactics.
ATPs are generally stand-alone tools that implement a (semi-)decision procedure for a given formal language or related set of languages.
For example, \emph{Vampire}~\cite{DBLP:conf/cav/KovacsV13} is an ATP focused primarily on first-order logic using superposition, which we used extensively in this project.  We also made extensive use of \emph{Prover9} and \emph{Mace}~\cite{prover9-mace4}.

ATPs are complex software that can contain bugs.
Instead of trusting ATP output, we used proof certificates, which many ATPs can produce, to reconstruct proofs in \emph{Lean}.
The details of proof reconstruction depend on the form of the proof certificate produced by the ATP.
We expand on this in \Cref{sec:proof-reconstruction}.

Tactics in \emph{Lean}, on the other hand, are meta-programs~\cite{DBLP:journals/pacmpl/EbnerURAM17} that build proofs.
In other words, they essentially take \emph{Lean} code as input and produce \emph{Lean} code as output.
In this manner, they look like another keyword in the language, and are tightly integrated by producing proofs directly.
Under the hood, their implementation can be arbitrarily complex, from syntactic sugar to full decision procedures.
The \texttt{duper} tactic~\cite{DBLP:conf/itp/CluneQBA24}, for example, implements a superposition calculus, similar to \emph{Vampire}'s, but for dependent types --- \emph{Lean}'s underlying logical foundation.

In the rest of this section we describe the different proof automation techniques used in this project.
We first discuss the different proof methods used: primarily superposition and equational reasoning, we then discuss the integration in \emph{Lean}, and finally we report some basic empirical results from this project.

\subsection{Proof techniques}

The main two families of ATPs and tactics we used are based on superposition/saturation and equational reasoning.
In this context we also include SMT solvers, which combine specific decision procedures for theories, like congruence closure for equational reasoning, with satisfiability (SAT) solving \cite{deMoura-Bjorner-2009}.
Finally, we also used \texttt{aesop}~\cite{DBLP:conf/cpp/LimpergF23}, which implements a version of tableau search.
This was used mainly to help specific constructions in refutations, and is not specific to proving or disproving magma implications in this sense.
We describe our use of \texttt{aesop} in \Cref{sec:proof-reconstruction} below.

\textbf{Saturation.}
Most of the ATPs used extensively in this project rely primarily on saturation procedures in the superposition calculus.
For example, this is the case for \emph{Vampire}~\cite{DBLP:conf/cav/KovacsV13}.\footnote{See also~\cite{DBLP:journals/cacm/BentkampBNTVW23} for a gentler exposition.}
The core idea of these provers is that they take a set of assumptions and a conjecture, expressed in --- say --- first-order logic.
The conjecture is negated and added to the set of assumptions, which are all put into a normal form.
The ATP then tries to refute the negation by applying rules of an underlying calculus, until a proof of false (a contradiction) is derived.
In this case, the conjecture was (classically) true, and the ATP has found a proof by contradiction, often called a ``refutation'' or ``saturation'' proof.

The underlying calculi vary from system to system, but they often have a variant of a resolution clause of the form:
\[\infer{C \lor D}{C \lor L \quad D \lor \neg L} \]
This can be read as $C \lor L$ with $D \lor \neg L$ implies $C \lor D$, where $C, D, L$ are formulas in e.g. first-order logic.
Superposition calculi have a variant of this rule that deals with equality directly, and thus are more efficient at reasoning about equality.

In this project we used \emph{Vampire}~\cite{DBLP:conf/cav/KovacsV13}, \emph{Duper}~\cite{DBLP:conf/itp/CluneQBA24} and \emph{Prover9} and \emph{Mace4}~\cite{prover9-mace4} which are all based on variants of saturation for proving.

\textbf{Equational Reasoning.} As already discussed in \Cref{canon-sec}, equational reasoning is a type of reasoning that is based\footnote{More precisely, one can formalize this reasoning using Birkhoff's five rules of inference (reflexive, symmetric, transitive, replacement, and substitution); see, e.g., \cite{burris}.} on equational logic and rewriting with congruence~\cite{term-rewriting}.
In general, an equational reasoning procedure takes a series of equations and tries to determine whether another equation can be deduced from it.
A core tool in equational reasoning are \emph{e-graphs}, a data structure used to represent congruence classes of terms.
By themselves, e-graphs provide an efficient means of implementing a decision procedure for congruence closure over ground equations (i.e. equations without variables).
Extensions to this procedure, for example by quantifier instantiation via e-matching \cite{DBLP:conf/cade/MouraB07}, also allow for a semi-decision procedure for congruence closure over non-ground equations.

SMT solvers like \emph{Z3}~\cite{DBLP:conf/tacas/MouraB08} use equational reasoning for deciding the theory of equality with uninterpreted functions~\cite{DBLP:series/txtcs/KroeningS16,DBLP:conf/cade/MouraB07}.
On the other hand, equality saturation~\cite{DBLP:journals/pacmpl/WillseyNWFTP21} uses e-graphs by extending congruence closure to a more controlled search, enabling optimization and conditional rewriting.
One of the main advantages of using equational reasoning to reason about implications of magma laws is that we get very explicit proofs: a proof that $l \models l'$ is given by a sequence of rewrites that starts at the left-hand side of $l'$ and arrives at the right-hand side through applications of $l$.

In this project we used \emph{Z3}~\cite{DBLP:conf/tacas/MouraB08}, \emph{Prover9} and \emph{Mace4}~\cite{prover9-mace4}, a custom ATP \emph{MagmaEgg} for magmas based on egg~\cite{DBLP:journals/pacmpl/WillseyNWFTP21}, and the \emph{Lean} \texttt{egg} tactic~\cite{DBLP:journals/pacmpl/KoehlerGBGTS24,rossel2024equality}, which all work with equational logic. We have also reasoned with manual (custom written) heuristics about simple rewrites.

\subsection{Integration of automation procedures}
\label{sec:proof-reconstruction}

While ATPs are very useful for solving theorems in this project, they do not integrate with \emph{Lean} out of the box.
ATPs may produce unsound proofs, or worse, derive incorrect results.
Thus, by default, theorems in \emph{Lean} cannot be proven by deferring to the result of an ATP.
Instead, the results of an ATP can be used to reconstruct a proof of the form required by \emph{Lean}.
% There are different approaches to proof reconstruction which we employ in this project.
Thus, in general, integration of ATPs requires two steps.
First, there is the invocation of the ATPs by translating the problem from \emph{Lean} into the languages and logics they use.
And second, there is the reconstruction of the ATPs' results as a (persistent) \emph{Lean} proof.
These two aspects present different challenges, and require different strategies, depending mostly on the kind of proof strategy the ATP uses.

More generally, we have observed that there are multiple ways of integrating decisions procedures within \emph{Lean}, with different levels of integration.

\begin{enumerate}
    \item Using a \emph{Lean} tactic, which calls a decision procedure written in \emph{Lean} (like \texttt{aesop} or \texttt{duper}).
    \item\label{inter} Using a \emph{Lean} tactic, which calls an existing (external) ATP and reconstructs a proof term from the ATP's result (like \texttt{bv\_decide} or \texttt{egg}).
    \item\label{external} Using an external script which calls an existing ATP and generates a source file \texttt{.lean} which captures the result explicitly.
\end{enumerate}

This project primarily used the least integrated approach, Option \ref{external}, as it was fastest to implement and imposed no additional technical requirements on other contributors.
The matter of technical requirements caused problems, for example, when integrating the \texttt{egg} tactic (Option \ref{inter}) as it initially expected certain software on the user's machine.
Such trade-offs between Option \ref{inter} and Option \ref{external} are, however, mutual, as the higher upfront cost of integrating a proof tactic in Option \ref{inter} makes the decision procedure easier to use than with Option \ref{external}.
Additionally, Option \ref{inter} can benefit from \emph{Lean}'s meta-programming capabilities when encoding the problem for use with an ATP, and when reconstructing a \emph{Lean} proof from the result.

\textbf{Proof Reconstruction.}

The relative simplicity of the objects used in this project benefit the implementations of proof reconstruction.
By focusing on the given problem domain, difficult reconstruction issues, like complex dependent types, could be ignored.

For saturation proofs with \emph{Vampire}, we implemented analogs of the \emph{superpose}, \emph{resolve}, and \emph{subsumption} steps in \emph{Lean}.
Proofs can then be reconstructed as sequences of these steps (and additional technicalities) as shown in Figure~\ref{fig:vampire-example}.

\begin{figure}
  \centering
  \includegraphics[width=\textwidth]{vampire-example.png}
  \caption{Example of a proof reconstructed from output of \emph{Vampire}. Note how the proof proceeds by contradiction and uses the \texttt{superpose} and \texttt{subsumption} steps implemented in \emph{Lean}.}
  \label{fig:vampire-example}
\end{figure}
% \todo{Use listings or minted for this code block.}

% \todo{Explain how Prover9 and Mace4 were used.}

For equational proofs from external provers, like \emph{MagmaEgg}, we also used a tailored version of reconstruction.
Specifically, the \emph{MagmaEgg} implementation turns \emph{explanations} \cite{nieuwenhuis2005proof} from \emph{egg} into \emph{Lean} proofs by simple applications of the defining properties of equality as shown in Figure~\ref{fig:magma-egg-example}.

\begin{figure}
  \centering
  \includegraphics[width=\textwidth]{magma-egg-example.png}
  % SOURCE: https://github.com/teorth/equational_theories/blob/main/equational_theories/Generated/MagmaEgg/small/_005.lean
  \caption{Example of a proof reconstructed by \emph{MagmaEgg}. Note the proof only uses reflexivity, symmetry, transitivity, and congruence of equality.}
  \label{fig:magma-egg-example}
\end{figure}
% \todo{Use listings or minted for this code block.}

In the case of the \texttt{egg} tactic, which also reconstructs proofs from \emph{egg} explanations, the proof could be converted into a more human-readable form by using the \texttt{calcify}\footnote{\url{https://github.com/nomeata/lean-calcify}} tactic, as shown in Figure~\ref{fig:egg-example}

\begin{figure}
  \centering
  \includegraphics[width=\textwidth]{egg-example.png}
  \caption{Example of the \texttt{egg} tactic reconstructing a proof in human-readable form with the help of \texttt{calcify} (invoked by the special syntax \texttt{egg?}).}
  \label{fig:egg-example}
\end{figure}

\textbf{Semi-Automated Counterexample Guidance.}  Another use of ATPs has been in a semi-automatic fashion, to find counterexamples.
The general strategy was to use ATPs to find counterexamples to implications by building magmas iteratively.
If we want to build a counterexample to $l \models l'$, we want to construct a magma where $l$ holds but $l'$ does not.
In this method, we iteratively strengthen a construction with additional hypotheses, and use the ATP to check whether these hypotheses are not too strong (to imply $l'$) or unsound (to disallow $l$).

% \TODO{this should also be expanded more, at least with references to some of the constructions in other chapters.}

While equational reasoning can also be used in a semi-automatic fashion to prove equations~\cite{DBLP:journals/pacmpl/KoehlerGBGTS24}, the positive implications in the main implication graph of project were all simple enough that we did not need a semi-automatic approach for them.

% \TODO{discuss guided search in the finite implications or the Higman-Neumann work Jose Brox has done.}

% \TODO{maybe add a screenshot here of the workflow of using a seed to find counterexamples with \emph{Prover9} or \emph{Vampire}?}

% \subsection{Empirical results}

% Finally, we report some empirical results from use of ATPs for this project, in terms of performance.
% This section is not intended to be a careful evaluation and benchmark comparison of the different ATPs; instead, we present our work here as a more informal ``field report'' documenting our experiences.
% In particular, we do not draw firm conclusions about the overall capabilities\footnote{One reason for this is that different ATPs were deployed at different stages of the project.  In particular, the later ATP runs were performed in an environment when a large fraction of the implications had already been settled, and only a small remainder set was tested by our project.  The current set of ATP-generated formalized implications has also been subject to a number of reductions to optimize compilation time, so we would caution against reading too much into the raw number of such formalizations in the codebase.} of the different ATPs.
% Rather, this serves as a use-case documenting the experience of (mostly) novice users.

% \TODO{throw a couple of "benchmarking" tables for the same ATP with different parameters and for different ATPs, talk about some relative gains in time (changing parameters we saw a 500 times speedup on this particular problem), etc. This is knowledge I think we have gained to some extent, and certainly I would have been glad to receive this kind of hints before we started!''.  Then leave it as an interesting open problem to properly develop and measure benchmarks for ATPs based on this project.}

% {\bf Any comparative study of semi-automated methods with fully automated ones? In principle, the semi-automated approach could be more automated using a script or "agent" to call various theorem provers. See \href{https://leanprover.zulipchat.com/#narrow/stream/458659-Equational/topic/A.20magma.20of.20order.20.3C.2013.20-.20for.20Equation2531.3F}{this discussion}}


% {\bf See \href{https://leanprover.zulipchat.com/#narrow/channel/458659-Equational/topic/1516.20-.3E.20255/near/481547543}{this discussion} on the value of using different ATPs and setting run time parameters etc. at different values.}

% {\bf What are the hardest implications to prove?  See \href{https://leanprover.zulipchat.com/#narrow/channel/458659-Equational/topic/What.20are.20the.20hardest.20positive.20implications.20for.20an.20ATP.3F}{this discussion}.}

\section{Implications for Finite Magmas}\label{austin-sec}

\TODO{Expand this sketch}

{\bf
Recap discussion from \url{https://leanprover.zulipchat.com/#narrow/channel/458659-Equational/topic/Austin.20pairs}
}

% \section{Order 5 laws}\label{order-5}

\note{TODO: report on laws on order 5}

\section{Higman--Neumann laws}\label{higman-neumann}

\TODO{report on Higman--Neumann laws}

\section{AI and Machine Learning Contributions}\label{ml-sec}

As discussed in \Cref{automated-sec}, the ETP made extensive use of automated theorem provers in completing the primary goal of determining and then formalizing all the implications between the specified equational laws.  In contrast, we were only able to utilize modern large language models (LLMs) in a fairly limited fashion.  Such models were useful in writing initial code for the graphical user interfaces discussed in \Cref{gui-sec}, as well as performing some code autocompletion (using tools such as \emph{Github Copilot}) when formalizing an informal proof in \emph{Lean}.  In one instance, \emph{ChatGPT} was used\footnote{\url{https://chatgpt.com/share/670ce7db-8a44-800d-a5dc-8462c12eca3b}} to guess a complete rewriting system for the law $\x \op ((y \op y) \op \z) \formaleq x \op y$ \eqref{eq1659} which could then be formally verified, thus resolving all implications from this equation. However, in most of the difficult implications that resisted automated approaches, we found that LLMs did not provide useful suggestions beyond what the human participants could already propose.

On the other hand, we found that machine learning (ML) methods showed some promise of being able to heuristically predict the truth value of portions of implication graph, as we shall now discuss\footnote{For some discussion of other ML experiments performed during the project, see \url{https://leanprover.zulipchat.com/\#narrow/channel/458659-Equational/topic/Machine.20learning.2C.20first.20results}.}.

\subsection{Graph ML: Directed link prediction on the implication graph}
\label{sec:dlp}

We experimented with various Graph Neural Network (GNN) autoencoders to predict missing edges,
providing a way to estimate the truth value of unproven implications. To assess these models,
we defined three test sets focusing on edge existence, directionality, and bidirectionality.
The results give insight into how these models handle dense, directed graphs like ours.
A more detailed report follows below.


\subsubsection{Motivation}

Directed Link Prediction~\cite{Kipf2016} is a method enabling machine learning and deep learning
models to predict missing edges in a directed graph. For our implication graph, this translates to
predicting the truth values of unproven implications. This task serves as a necessary first step
for advancing in the following directions:

\begin{enumerate}
    \item \textbf{Reasoning over mathematical knowledge graphs:} Recent advancements allow language
    models to integrate information from multiple modalities. For example,
    \cite{Zhang2022, Yasunaga2022} share information between corresponding layers of Language Models (LMs)
    and Graph Neural Networks (GNNs), enabling simultaneous learning from text corpora and graph-structured
    data within the same expert domain. By leveraging both modalities, the language model can better
    \emph{structure} its knowledge and respond to complex queries. This dual learning process combines
    masked language modeling for text with link prediction on the graph, highlighting the importance
    of link prediction for robust reasoning.
    \item \textbf{Higher-order implication graphs:} Our implication graph currently represents only
    implications of the form $p \implies q$, not more complex ones like $(p \land r) \implies q$.
    Extending to such higher-order edges would likely involve connecting sets of nodes, thereby
    requiring hypergraph representations. For a systematic overview, see \cite{Kivela2014}.
    While specific hypergraph neural architectures exist~\cite{Feng2019}, we believe it is still
    conceptually important to apply Directed Link Prediction to simpler implication graphs first,
    providing insights and guiding principles that can anticipate challenges in higher-order graph
    representation learning.
\end{enumerate}

\subsubsection{Data}

We used \href{https://github.com/teorth/equational_theories/blob/main/data/2024-10-20-edge_list.csv.zip}{this edge list},
generated on October 20, 2024, with the following commands:

\begin{verbatim}
    lake exe extract_implications outcomes > data/tmp/outcomes.json
    scripts/generate_edgelist_csv.py
\end{verbatim}

The structure of the implication graph is summarized below:

\begin{verbatim}
    graph_summary = {
        "total_nodes": 4694,
        "total_directed_edges": 8178279,
        "edge_density_percentage": 37,  # % of possible edges that exist
        "bidirectional_edges": 2475610,
        "bidirectional_percentage": 30  # % of all edges that are bidirectional
    }
\end{verbatim}

Below is a summary of the edge types in the graph:

\begin{verbatim}
    edge_counts = {
        "explicit_conjecture_false": 92,
        "explicit_proof_false": 582316,
        "explicit_proof_true": 10657,
        "implicit_conjecture_false": 142,
        "implicit_proof_false": 13272681,
        "implicit_proof_true": 8167622,
        "unknown": 126
    }
\end{verbatim}

Edges are labeled according to the following scheme:

\begin{verbatim}
    edge_labels = {
        "implicit_proof_true": 1,
        "explicit_proof_false": 0,
        "implicit_proof_false": 0,
        "explicit_proof_true": 1,
        "explicit_conjecture_false": 0,
        "implicit_conjecture_false": 0,
        "unknown": 0
    }
\end{verbatim}

The \texttt{unknown} class contains a very small number of edges (126), so their impact on the
training phase is expected to be negligible. Future approaches might address this class by
excluding \texttt{unknown} edges from the training set.

\subsubsection{Methods}

Consider a directed graph $G = (V, E)$ where $E = \{(u,v) \mid u, v \in V\}$ is the edge set and
$|V| = n$. We assume that each node is associated with a feature vector, resulting in an
$X \in \mathbb{R}^{n \times f}$ feature matrix.

We define the existing edges $(a, b) \in E$ as \emph{positives} and the non-existing edges
$(c, d) \notin E$ as \emph{negatives}.

Intuitively, performing Directed Link Prediction (DLP) on $G$ involves randomly splitting $E$
into three disjoint sets: $E_{\text{train}}$, $E_{\text{val}}$, and $E_{\text{test}}$, such that:

\begin{itemize}
    \item $E_{\text{train}}$ is the training set,
    \item $E_{\text{val}}$ is the validation set,
    \item $E_{\text{test}}$ is the test set,
    \item and $E = E_{\text{train}} \dot{\cup} E_{\text{val}} \dot{\cup} E_{\text{test}}$.
\end{itemize}

The model then learns from $G_{\text{train}} = (V, E_{\text{train}})$ to map the topological and
feature-related information of two nodes $u$ and $v$ to a probability $p_{uv}$ that
$(u, v) \in E_{\text{test}}$.

However, this setup presents two key issues among others:

\begin{enumerate}
    \item The model learns only from \emph{positives}, so it cannot recognize \emph{negatives}.
    \item The model is evaluated only on \emph{positives}, preventing us from measuring its ability
    to identify \emph{negatives}.
\end{enumerate}

To address these limitations, we adopted the setup proposed in~\cite{Salha2019}.
Specifically, we redefined $E_{\text{train}}$ to $E_{\text{train}}^p$ (\emph{positives}) and introduced:

\[
E_{\text{train}} = E_{\text{train}}^p \dot{\cup} E_{\text{train}}^n
\]

where $E_{\text{train}}^n$ includes all possible \emph{negatives} in $G_{\text{train}} = (V, E_{\text{train}}^p)$.
The model is now required to predict the non-existence of edges in $E_{\text{train}}^n$.

Similarly, if we redefine $E_{\text{test}}$ as follows:

\[
E_{\text{test}} = E_{\text{test}}^p \dot{\cup} E_{\text{test}}^n
\]

where $E_{\text{test}}^n$ is a \emph{random} sample of \emph{negatives}, the model's evaluation
would fail to capture two crucial aspects:

\begin{enumerate}
    \item The model's ability to distinguish $(u,v)$ from $(v,u)$ for all $(u,v) \in E_{\text{test}}^p$.
    \item The model's ability to identify bi-implications.
\end{enumerate}

These limitations arise from the random selection of negative edges in $E_{\text{test}}^n$.
To address this, we define three distinct test sets: $E_{\text{test}}^G$, $E_{\text{test}}^D$,
and $E_{\text{test}}^B$, to evaluate different facets of the model’s performance:

\begin{itemize}
    \item \textbf{General Test Set} ($E_{\text{test}}^G$):
    Here, $E_{\text{test}} = E_{\text{test}}^p \dot{\cup} E_{\text{test}}^n$, where $E_{\text{test}}^n$
    is a random sample of non-existent edges with the same cardinality as $E_{\text{test}}^p$.
    This set assesses the model's ability to detect the presence of edges, regardless of direction.
    A model that cannot distinguish edge direction may still perform well on this set, highlighting
    the need for the following two additional test sets.
    \item \textbf{Directional Test Set} ($E_{\text{test}}^D$):
    Defined as $E_{\text{test}}^{\text{up}} \dot{\cup} \tilde{E}_{\text{test}}^{\text{up}}$,
    where $E_{\text{test}}^{\text{up}}$ consists of unidirectional edges in $E_{\text{test}}^p$,
    and $\tilde{E}_{\text{test}}^{\text{up}}$ contains their reverses (negatives by construction).
    This set evaluates the model's ability to recognize edge direction, making it suitable for
    assessing direction-sensitive models.
    \item \textbf{Bidirectional Test Set} ($E_{\text{test}}^B$):
    Defined as $E_{\text{test}}^{\text{bp}} \dot{\cup} E_{\text{B}}^{\text{n}}$,
    where $E_{\text{test}}^{\text{bp}}$ contains one direction of all bidirectional edges in $E_{\text{test}}^p$,
    and $E_{\text{B}}^{\text{n}} \subset \tilde{E}$ includes a subset of their reverses.
    This set evaluates the model's ability to identify bi-implications, as each edge in $E_{\text{test}}^B$
    has a reverse that is positive, but only half are bidirectional in practice.
\end{itemize}

We tested the following models:

\begin{itemize}
    \item \textbf{GAE}~\cite{Kipf2016}
    \item \textbf{Gravity-GAE}~\cite{Salha2019}
    \item \textbf{Source/Target-GAE}~\cite{Salha2019}
    \item \textbf{DiGAE}~\cite{Kollias2022}
    \item \textbf{MagNet}~\cite{Zhang2021}
\end{itemize}

All these models are graph-based autoencoders with distinct encoder-decoder architectures.
Notably, GAE is the only model structurally unable to differentiate edge directions.
Each model outputs the probability that an ordered pair of nodes has a directed edge between them,
with nodes represented using one-hot encodings as features.

We trained the models using Binary Cross Entropy as the loss function, balancing the contribution
of positive and negative edges through re-weighting. On the \emph{General} test set, we evaluated
the following metrics:

\begin{itemize}
    \item \textbf{AUC} (Area Under the ROC Curve): Measures the probability that the model ranks a
    random positive edge higher than a random negative edge. Higher values indicate better
    discrimination between positive and negative edges.
    \item \textbf{AUPRC} (Area Under Precision-Recall Curve): Assesses model performance,
    particularly in cases of class imbalance. AUPRC is more robust to imbalanced data than AUC.
    \item \textbf{Hits@K}: Evaluates the fraction of times a positive edge is ranked within the
    top $K$ positions among personalized negative samples~\cite{Li2023}. Briefly, given a positive
    edge, its $M$ personalized negative samples are $M$ negative edges with the same head but
    different tails. We calculate Hits@K for $K = 1, 3, 10$ to assess ranking quality, and
    set $M = 100$. Therefore, Hits@K = 0.1 means that on average, the model ranks a positive
    edge within the highest-ranked $K$ personalized negatives $10\%$ of the time.
    \item \textbf{MRR} (Mean Reciprocal Rank): Computes the average reciprocal rank of positive
    edges among their personalized negative samples~\cite{Li2023} (the same as those used for Hits@K)
    providing an overall measure of ranking accuracy. For instance, $MRR = 0.1$ means that on average,
    the model ranks a positive edge as $10^{\text{th}}$ among $M$ personalized negative samples.
\end{itemize}

Each metric ranges from 0 to 1, with higher values reflecting improved performance.
Based on prior work, we expect AUC and AUPRC scores to approach 1, while MRR and Hits@K often
yield values around 0.15 for similar undirected tasks~\cite{Li2023}. \emph{Directional} and
\emph{Bidirectional} performances were evaluated using only AUC and AUPRC, since Hits@K and MRR
are hardly definable in those scenarios. The number of training epochs was optimized through
Early Stopping on the \emph{General} validation set performance (given by the sum of AUC and AUPRC).

\subsubsection{Results}

The results below represent average performance over six random splits of $E_{\text{train}}$,
$E_{\text{val}}$, and $E_{\text{test}}$ while keeping the model's seed fixed for fair comparison.
The \emph{training / validation / test} split proportions are:

\begin{itemize}
    \item $85 / 5 / 10$ for unidirectional edges,
    \item $65 / 15 / 30$ for bidirectional edges.
\end{itemize}

\begin{table}[h]
\centering
\begin{tabular}{lcccccccccc}
\hline
\textbf{Model} & \textbf{G\_ROC\_AUC} & \textbf{G\_AUPRC} & \textbf{G\_Hits@1} & \textbf{G\_Hits@3} & \textbf{G\_Hits@10} & \textbf{G\_MRR} & \textbf{D\_ROC\_AUC} & \textbf{D\_AUPRC} & \textbf{B\_ROC\_AUC} & \textbf{B\_AUPRC} \\
\hline
\texttt{gae}              & $0.8484 \pm 9 \times 10^{-4}$  & $0.8558 \pm 6 \times 10^{-4}$   & $6 \times 10^{-5} \pm 4 \times 10^{-5}$ & $6 \times 10^{-5} \pm 4 \times 10^{-5}$ & $0.0001 \pm 4 \times 10^{-5}$ & $0.0165 \pm 2 \times 10^{-4}$ & $0.5 \pm 0$          & $0.5 \pm 0$          & $0.941 \pm 5 \times 10^{-3}$    & $0.964 \pm 3 \times 10^{-3}$    \\
\texttt{gravity\_gae}      & $0.9806 \pm 3 \times 10^{-4}$  & $0.9753 \pm 4 \times 10^{-4}$   & $0.069 \pm 6 \times 10^{-3}$ & $0.101 \pm 5 \times 10^{-3}$ & $0.17 \pm 3 \times 10^{-3}$   & $0.112 \pm 5 \times 10^{-3}$  & $0.9958 \pm 1 \times 10^{-4}$   & $0.9874 \pm 2 \times 10^{-4}$   & $0.99717 \pm 4 \times 10^{-5}$  & $0.99431 \pm 7 \times 10^{-5}$  \\
\texttt{sourcetarget\_gae} & $0.99976 \pm 1 \times 10^{-5}$ & $0.999736 \pm 8 \times 10^{-6}$ & $0.077 \pm 4 \times 10^{-3}$ & $0.147 \pm 7 \times 10^{-3}$ & $0.279 \pm 9 \times 10^{-3}$  & $0.152 \pm 5 \times 10^{-3}$  & $0.999982 \pm 1 \times 10^{-6}$ & $0.999983 \pm 1 \times 10^{-6}$ & $0.999989 \pm 2 \times 10^{-6}$ & $0.999987 \pm 3 \times 10^{-6}$ \\
\texttt{mlp\_gae}          & $0.99315 \pm 1 \times 10^{-5}$ & $0.99409 \pm 1 \times 10^{-5}$  & $0.181 \pm 7 \times 10^{-3}$ & $0.299 \pm 7 \times 10^{-3}$ & $0.53 \pm 2 \times 10^{-3}$   & $0.289 \pm 6 \times 10^{-3}$  & $0.99671 \pm 2 \times 10^{-5}$  & $0.9973 \pm 1 \times 10^{-5}$   & $0.99692 \pm 2 \times 10^{-5}$  & $0.99736 \pm 2 \times 10^{-5}$  \\
\texttt{digae}            & $0.9978 \pm 3 \times 10^{-4}$  & $0.998 \pm 3 \times 10^{-4}$    & $0.035 \pm 6 \times 10^{-3}$ & $0.068 \pm 1 \times 10^{-2}$ & $0.18 \pm 2 \times 10^{-2}$   & $0.091 \pm 1 \times 10^{-2}$  & $0.9991 \pm 2 \times 10^{-4}$   & $0.9993 \pm 2 \times 10^{-4}$   & $0.9994 \pm 1 \times 10^{-4}$   & $0.9995 \pm 1 \times 10^{-4}$   \\
\texttt{magnet}           & $0.989 \pm 1 \times 10^{-4}$   & $0.99076 \pm 3 \times 10^{-5}$  & $0.151 \pm 1 \times 10^{-2}$ & $0.26 \pm 2 \times 10^{-2}$  & $0.38 \pm 2 \times 10^{-2}$   & $0.24 \pm 1 \times 10^{-2}$   & $0.9962 \pm 1 \times 10^{-3}$   & $0.9969 \pm 6 \times 10^{-4}$   & $0.9976 \pm 4 \times 10^{-4}$   & $0.9979 \pm 2 \times 10^{-4}$   \\
\hline
\end{tabular}
\caption{Results for various graph autoencoder models.}
\end{table}

\subsubsection{Discussion}

We observe consistently high \emph{General} AUC and AUPRC scores, close to 1.
These high values are expected, as similar neural architectures have demonstrated strong performance
in graphs of comparable size~\cite{Kipf2016}. The high ratio of existing to non-existing edges in the
implication graph (approximately 37\%) likely contributes to the near-perfect \emph{General} AUC and
AUPRC scores. For context, benchmark datasets such as Cora and Citeseer
(e.g., \href{https://github.com/deezer/gravity_graph_autoencoders/tree/master/data}{directed}
and \href{https://pytorch-geometric.readthedocs.io/en/latest/generated/torch_geometric.datasets.Planetoid.html}{undirected})
contain fewer than 1\% of all possible edges.

Interestingly, the GAE model, though structurally unable to distinguish edge direction, performs well
on the \emph{General} task (if we consider AUC and AUPRC only). This experimentally confirms the need
for including \emph{Directional} and \emph{Bidirectional} test sets, which allow comprehensive
evaluation across all facets of Directed Link Prediction (DLP).

All other models demonstrate high AUC and AUPRC scores across the \emph{General}, \emph{Directional},
and \emph{Bidirectional} test sets, indicating strong predictive capabilities even when accounting for
directionality and bidirectionality.

Notably, the \texttt{mlp\_gae} and \texttt{magnet} models also achieve competitive scores in MRR and
Hits@K metrics.

\subsubsection{Conclusions}

We evaluated the performance of six different graph autoencoders on a Directed Link Prediction (DLP) task.
By adopting a multi-task evaluation framework, we assessed the models comprehensively across various
aspects of DLP. All models demonstrated strong performance on AUC and AUPRC metrics, with some also
achieving high scores on MRR and Hits@K.

Node features were represented using one-hot encodings, meaning that the models received no explicit
information about the equations represented by the nodes. Instead, they relied entirely on the
topological structure encoded during training. This approach aligns with previous research suggesting
that one-hot encodings can promote asymmetric embeddings~\cite{Salha2019}. However, future experiments
could explore alternative representations, such as encoding the equations with text-based embeddings
like Word2Vec, to potentially enhance the models' understanding of the nodes' semantic content.

In summary, our findings highlight the adaptability and robustness of graph autoencoders for DLP
tasks in dense, directed graphs. This approach not only demonstrates robustness in predicting directed
links but also suggests promising potential for future improvements through enhanced feature
representations, thereby advancing the capabilities of link prediction in complex mathematical
knowledge graphs.

\section{User Interfaces}

A number of custom web applications were developed as part of the ETP. While many past Lean formalization projects have primarily relied on the Lean blueprint tool to organize tasks and track progress, the large volume of (transitive) implications tracked by the ETP, along with the research-oriented nature of the project, necessitated the development of custom tools to complement the blueprint tool. These web applications also made information more accessible to project participants and other interested parties, including those unfamiliar with Lean or the custom software developed for the project. The project features four primary interfaces:

\begin{enumerate}
  \item The \textbf{ETP dashboard}\footnote{\url{https://teorth.github.io/equational_theories/dashboard/}} displays the high-level overview of the project: the total number of resolved, conjectured, and unknown implications for the general and finite implication graphs. The dashboard also includes links to other tools, data, and visualizations about the implication graphs.
  \item The \textbf{Equation Explorer}\footnote{\url{https://teorth.github.io/equational_theories/implications/}} is the primary tool to navigate the implication graph. For a given equation, it display its inbound and outbound implications, as well as other members of its equivalence class. The explorer allows navigating either the general or finite implication graphs. The explorer also features custom commentary for a given equation (when available), serving as a repository for information and links. It also links to Graphiti visualizations and an example of its smallest satisfying magma, if one exists. Figure~\ref{fig:screenshot-equation-explorer} shows an example view of the explorer.
  \item \textbf{Graphiti}\footnote{\url{https://teorth.github.io/equational_theories/graphiti/}} visualizes the implication graph as a Hasse diagram, where downward edges represent subset relationships, and upward edges represent implications. Equivalence classes are collapsed into single nodes for clarity. Graphiti supports search parameters to visualize specific subsets of the graph. It can also display the entire implication graph, though the complete graph is large and challenging to navigate. Figure~\ref{fig:854-like} is an example of a Graphiti visualization.
  \item The \textbf{Finite Magma Explorer}\footnote{\url{https://teorth.github.io/equational_theories/fme/}} tests which equations a given finite magma satisfies or fails to satisfy. Users input finite magmas as Cayley tables. The tool is aware of the finite implication graph, so if an input magma witnesses an unknown refutation, it notifies the user and provides instructions for contributing the result to the GitHub repository.
\end{enumerate}

\begin{figure}
  \centering
  \includegraphics[width=0.85\textwidth]{GUI-equation-explorer.png}
  \caption{An example of the information displayed by the Equation Explorer for a specific equation.}
  \label{fig:screenshot-equation-explorer}
\end{figure}

The data for these tools is extracted directly from the Lean-formalized proofs in the project's GitHub repository, ensuring it always faithfully reflects the current state of progress. Additionally, the data is automatically updated with each code change using continuous integration (CI), eliminating the need for manual updates.

\section{Data Management}

\note{TODO: expand this sketch}

Describe how data was handled during the project and how it will be managed going forward.


\section{Conclusions and future directions}

This project successfully demonstrated that large-scale explorations of a space of mathematical statements (in this case, the implications or non-implications between selected equational laws) can be crowdsourced using modern collaboration platforms and proof assistants.  No single tool or method was able to study the entirety of this space, and many informal proofs generated contained non-trivial errors; but there were multiple techniques that could treat significant portions of the space, and through a collaborative effort combined with the proof validation provided by \emph{Lean}, one could synthesize these partial and fallible contributions into a complete and validated description of the entire implication graph.  While this particular graph was a comparatively simple structure to analyze, we believe that this paradigm could also serve as a model for future projects devoted to exploring more sophisticated large-scale mathematical structures.

Several factors appeared to be helpful in ensuring the success of the project, including the following:
\begin{itemize}
\item \textbf{A clearly stated primary goal, with an end condition and precise numerical metrics to measure partial completion.}  From the outset, there was a specific goal to attain, namely to completely determine and then formalize the implication graph on the original set of $4694$ laws.  Progress towards that goal could be measured by a number of metrics, such as the number of implications that were conjectured but unformalized, or not conjectured at all.   Such metrics allowed participants to see how partial contributions, such as formalizing a certain subset of implications, advanced the project directly towards its primary goal.  This is not to say that all activity was devoted solely towards this primary goal, but it did provide a coherent focus to help guide and motivate other secondary activities.
\item \textbf{A highly modular project}.  It was possible for any given coauthor to work on a small subset of implications and focus on a single proof technique, without needing to understand or rely upon other contributions to the project.  This allowed the work to be both parallelized and decentralized; many contributors launched their own investigations broadly within the framework of the project, without needing centralized approval or coordination.
\item \textbf{Low levels of required mathematical and formal prerequisites}.  The problems considered in the project did not require advanced mathematical knowledge (beyond a general familiarity with abstract algebra), nor a sophisticated understanding of formal proof assistants.  This permitted contributions from a broad spectrum of participants, including those without a graduate mathematical training, as well as mathematicians with no experience in proof formalization.  At a technical level, it also meant that formalization of proofs into \emph{Lean} could be done immediately once certain base definitions (such as \texttt{Magma}) were constructed.  This can be compared for instance with the recent formalization of the Polynomial Freiman--Ruzsa conjecture\footnote{\url{https://github.com/teorth/pfr}}, in which significant effort was expended in the first few days to settle on a suitable framework to formalize the mathematics of Shannon entropy.  While some more sophisticated formal structures (such as the syntactic description of laws as pairs of words in a \texttt{FreeMagma}) were later introduced in the project, it was relatively straightforward to refactor previously written code to be compatible with these structures as they were incorporated into the project.
\item \textbf{Variable levels of difficulty, and the amenability to partial progress.}  Traditional mathematics projects generally involve a small number of extremely hard problems, with incomplete progress on these problems being difficult to convert into clean partial results.  In contrast, the ETP studied a large number of problems with a very broad range of difficulty, so that even if a given proof strategy did not work for a given implication, it could be the case that there was some class of easier implications for which the strategy was successful.  This allowed for a means to validate such ideas, and allowed the project to build up a useful and diverse toolbox of proof techniques which became increasingly necessary to handle the final and most difficult implications in the project.  It also created a dynamic in which the project initially focused on easy techniques to resolve a significant fraction of the implications, gradually transitioning into more sophisticated methods that focused on a much smaller number of outstanding implications that had proven resistant (or even ``immune'') to all easier approaches.
\item \textbf{Centralized and standardized platforms for discussion, project management, and validation.}  While the project was decentralized at the level of the participant, there was a centralized location (a channel\footnote{\url{https://leanprover.zulipchat.com/\#narrow/channel/458659-Equational}} on the Lean Zulip) to discuss all aspects of the project, as well as a centralized repository\footnote{\url{https://github.com/teorth/equational_theories}} to track all contributions and outstanding issues, a centralized blueprint\footnote{\url{https://teorth.github.io/equational_theories/blueprint/}} to describe technical details of proofs to be formalized, and a single formal language (\emph{Lean}) to validate all contributions. A significant portion of the activity in the early stages of the project was devoted to setting out the standards and workflows for handling both the discussion and the contributions, in particular setting up a contributions page\footnote{\url{https://github.com/teorth/equational_theories/blob/main/CONTRIBUTING.md}} and adopting a code of conduct\footnote{\url{https://github.com/teorth/equational_theories/blob/main/CODE_OF_CONDUCT.md}}.  This gave some structure and predictability to what might otherwise be a chaotic effort.
\item \textbf{Development of custom visualization tools.}  As discussed in \Cref{sec:gui-sec}, several tools were developed (in part with AI assistance) to help visualize and navigate the implication graph while it was in a partial stage of development, allowing for participants to independently identify problems to work on, and to validate and use the contributions of other participants even before they were fully formalized.  For instance, a participant could propose a finite counterexample to an implication by posting a link to the magma in \emph{Finite Magma Explorer}, allowing for immediate validation of the counterexample, or use \emph{Equation Explorer} or \emph{Graphiti} to observe some interesting phenomenon in the implication graph that other participants could reproduce and study.
\item \textbf{Applicability of existing software tools.}  As described in \Cref{automated-sec}, many of the implications in the ETP were amenable to application of ``off-the-shelf'' automated theorem provers (ATPs); while some trial and error was needed to determine good choices of parameters, these tools could largely be applied directly to the project without extensive customization.  (However, the later transcription of ATP output into Lean was sometimes non-trivial.)
\item \textbf{Receptiveness to new techniques and tools.}  Crucially, the methods used to make progress on the project were not specified in advance, and contributions from participants with new ideas, techniques, or software tools that were not initially anticipated were welcomed.  For instance, the theory of canonizers (\Cref{canon-sec}) was not initially known to the first project participants, but was brought to the attention of the project by a later contributor.  Conversely, while there were hopes expressed early in the project that modern large language models (LLMs) could automatically generate many of the proofs required, it turned out in practice that other forms of automation, particularly ATPs, were significantly more effective at this task (at least if one restricted to publicly available LLMs), and the project largely moved away from the use of such LLMs (other than to help create the code for the visualization tools).
\end{itemize}

There are several mathematical and computational questions that could potentially be addressed in future work building upon the outcomes of ETP\@.
Here is a list of some possible such future directions.
\begin{enumerate}
  \item Does the law $\x \formaleq \y \op (\x \op ((\y \op \x) \op \y))$ \eqref{eq677} imply $\x \formaleq ((\x \op \x) \op \x) \op \x$ \eqref{eq255} for finite magmas, i.e., $\Eq{677} \modelsfin \Eq{255}$? This is the last remaining implication (up to duality) for finite magmas to be resolved.  A number of partial results on this problem may be found at \url{https://teorth.github.io/equational_theories/blueprint/677-chapter.html}.

  \item The ETP focused on determining relations $\E \models \E'$ between one law and another.  Could the same methods also systematically determine more complex logical relations, such as $\E_1 \wedge \E_2 \models \E_3$, for all laws $\E_1,\E_2,\E_3$ in a specified set?  This includes the question of implications between equational laws in semigroups (associative magmas).  One could also consider implications involving magma properties that are not equational laws, such as cancellability or existence of a unit element.

  \item Call an implication $\E_1 \models \E_2$ ``irreducible'' if there is no equational law $\E$ with $\E_1 \models \E \models \E_2$, other than those laws equivalent to either $\E_1$ or $\E_2$.  For instance, $\Eq{2} \models \Eq{4}$ is irreducible, since $\Eq{4}$ implies any law of the form $w \formaleq w'$ where the left-most variable of $w$ matches the left-most variable of $w'$. On the other hand, $\Eq{4}$ in conjunction with any law not of that form yields $\Eq{2}$.  Similar \emph{ad hoc} arguments can produce other irreducible implications, e.g., $\Eq{2} \models \Eq{n}$ for $n = 5, 895, 26302$.  Could one replicate the ETP to classify all stable implications among the same \num{4694} equations studied in this project?

  \item For a given finite non-implication $\E_1 \nmodelsfin \E_2$, are there bounds on the proportion of variable assignments for which~$\E_2$ holds, similarly to how in a finite group either all elements square to the neutral element, or at most $3/4$ of them do?
\end{enumerate}
Some other directions do not concern implications between laws, but may benefit from data generated by the ETP\@.
\begin{enumerate}[resume*]
  \item Does the law $\x \formaleq \y \op (\y \op (\y \op (\x \op (\z \op \y))))$ \eqref{eq5093} have any infinite models? In \cite{Kisielewicz2} it was shown that it has no non-trivial finite models, but the infinite model case was left as an open question.  A partial classification of laws of order~$5$ with infinite models but no finite models is given at \url{https://teorth.github.io/equational_theories/blueprint/order-5-austin-laws.html}.

  \item A key feature of finite magmas $\Magma$ is that they are surjunctive, in the sense that any definable map from $M$ to itself that is injective, is also surjective (or vice versa), where ``definable'' is with respect to the language of magmas.  Are there equational theories that admit surjunctive models, but yet do not have any non-trivial finite models?

  \item Are all finite weak central groupoids, namely magmas obeying $\x \formaleq (\y \op \x) \op (\x \op (\z \op \y))$ \eqref{eq1485}, necessarily of size $n^2$ or $2n^2$?  More generally, what is the spectrum of each law or conjunction of laws, and what are the possible asymptotics for the fine spectrum in terms of model size?

  \item How ``stable'' is a given law~$\E$?  For instance, if a finite magma satisfies a law~$\E$ some proportion $1-\eps$ of the time, with $\eps$~small, can the magma be perturbed into one that satisfies~$\E$ exactly?  Related to this is the question of whether a law~$\E$ is ``rigid'' or ``mutable'': is it possible to add an element or to make a small number of modifications to a magma satisfying~$\E$, in a way that still preserves~$\E$?  Such properties helped suggest whether certain magma construction techniques, such as modifying a base magma, were likely to be successful.

  \item For each law, can its free magma with one or more generators be described explicitly?

  \item Which laws admit an interesting theory of smooth magmas, analogous to Lie groups?
\end{enumerate}

\subsection{Miscellaneous remarks}

It is possible that the timing in which certain proof methods were introduced into the project created some opportunity costs.  For instance, by deploying automated theorem provers at an early stage, we might have settled some implications that had more interesting human-readable proofs that we missed.  Similarly, we developed some sophisticated theory for the equation $\Eq{854}$, such as \Cref{unique-factorization}, that is now superseded by finite counterexamples; but had the finite counterexamples been discovered first, we would not have found the theoretical arguments.  It may be productive for future work to revisit some portions of the implication graph and locate alternate proofs and methods.


\section*{Acknowledgments}

We are grateful to the many additional participants in the Equational Theories Project for their
numerous comments and encouragement, with particular thanks to Stanley Burris, Edward van de Meent and David Roberts. We warmly thank Michael Kinyon for generously sharing his expertise with \emph{Prover9–Mace4}, and we are likewise grateful to Laura Kovács, Márton Hajdu, Martin Suda, and Michael Rawson of the \emph{Vampire} development team for their helpful explanations regarding \emph{Vampire}'s options and functionality. Additionally, we note that Shreyas Srinivas is a doctoral student at the Saarbr\"{u}cken Graduate School for Computer Science.


\appendix

\section{Numbering system}\label{numbering-app}

In this section we record the numbering conventions we use for equational laws.

For this formal definition we use the natural numbers $0,1,2,\dots$ to represent and order indeterminate variables; however, in the main text, we use the symbols $x,y,z,w,u,v,r,s,t$ instead (and do not consider any laws with more than eight variables).

To define the ordering we use on equational laws, we first consider the case where there is a single indeterminate $\ast$.
We place a well-ordering on words $w,w'$ with a single indeterminate $\ast$ by declaring $w > w'$ if one of the following holds:
\begin{itemize}
    \item $w$ has a larger order than $w'$.
    \item $w = w_1 \op w_2$ and $w' = w'_1 \op w'_2$ have the same order $n \geq 1$ with $w_1 > w'_1$.
    \item $w = w_1 \op w_2$ and $w' = w'_1 \op w'_2$ have the same order $n \geq 1$ with $w_1 = w'_1$ and $w_2 > w'_2$.
\end{itemize}
Thus, for instance
$$ \ast < \ast \op \ast < \ast \op (\ast \op \ast) < (\ast \op \ast) \op \ast.$$

We similarly place a well-ordering on equational laws $w_1 \formaleq w_2$ with a single indeterminate $\ast$ by declaring $w_1 \formaleq w_2 > w'_1 \formaleq w'_2$ if one of the following holds:
as follows:
\begin{itemize}
\item  $w_1 \formaleq w_2$ has a longer order than $w'_1 \formaleq w'_2$.
\item If $w_1 \formaleq w_2$ has the same order as $w'_1 \formaleq w'_2$, and $w_1 > w'_1$.
\item If $w_1 \formaleq w_2$ has the same order as $w'_1 \formaleq w'_2$, $w_1 = w'_1$, and $w_2 > w'_2$.
\end{itemize}
Thus for instance
$$ (\ast \op \ast \formaleq \ast \op (\ast \op \ast)) < (\ast \op \ast \formaleq (\ast \op \ast) \op \ast).$$

Finally for equational laws with alphabet $x,y,z,w,u,v,r,s,t$, define the \emph{shape} of that law to be the law formed by replacing all indeterminates with $\ast$; for instance, the shape of \eqref{eq4512}, $(\ast \op \ast) \op \ast \formaleq \ast \op (\ast \op \ast)$, is $(\ast \op \ast) \op \ast \formaleq \ast \op (\ast \op \ast)$.  We then place a well-ordering $w_1 \formaleq w_2$ with indeterminates $x,y,z,w,u,v,r,s,t$ by declaring $w_1 \formaleq w_2 > w'_1 \formaleq w'_2$ if one of the following holds:
\begin{itemize}
\item The shape of $w_1 \formaleq w_2$ is greater than the shape of $w'_1 \formaleq w'_2$.
\item $w_1 \formaleq w_2$ and $w'_1 \formaleq w'_2$ have the same shape, and the string of variables appearing in $w_1 \formaleq w_2$ is lower in the lexicographical ordering (using $x < y < z < w < u < v < r < s < t$) than the corresponding string for $w'_1 \formaleq w'_2$.
\end{itemize}
Thus for instance any law of shape $\ast \op \ast \formaleq \ast \op (\ast \op \ast)$ is lower than any law of shape
$\ast \op \ast \formaleq (\ast \op \ast) \op \ast$.  Among the laws of shape $\ast \op \ast \formaleq \ast \op (\ast \op \ast)$, the lowest is $\x \op \x \formaleq \x \op (\x \op \x)$, which is less than (say) $\x \op \x \formaleq \y \op (\y \op \y)$, which is in turn less than $\x \op \y \formaleq \x \op (\x \op \x)$.

Two equational laws are equivalent if one can be obtained from another by some combination of relabeling the variables and applying the symmetric law $w_1 \formaleq w_2 \iff w_2 \formaleq w_1$.  For instance, $(0 \op 1) \op 2 \formaleq 1$ is equivalent to $0 \formaleq (1 \op 0) \op 2$.  We then replace every equational law with their minimal element in their equivalence class, which can be viewed as the \emph{normal form} for that law; for instance, the normal form of $(0 \op 1) \op 2 \formaleq 1$ would be $0 \formaleq (1 \op 0) \op 2$.  Finally, we eliminate any law of the form $w \formaleq w$ other than $0 \formaleq 0$.  We then number the remaining equations $E1, E2, \dots$.  For instance, $E1$ is the trivial law $0 \formaleq 0$, $E2$ is the constant law $0 \formaleq 1$, $E3$ is the idempotent law $0 \formaleq 0 \op 0$, and so forth.  Lists and code for generating these equations, or the equation number attached to a given equation, can be found in the ETP repository.

The number of equations in this list of order $n=0,1,2,\dots$ is given by
$$ 2, 5, 39, 364, 4284, 57882, 888365, \dots$$
(\url{https://oeis.org/A376640}).  The number can be computed to be
$$ C_{n+1} B_{n+2}/2$$
if $n$ is odd, $2$ if $n=0$, and
$$ (C_{n+1} B_{n+2}+ C_{n/2}(2D_{n+2}-B_{n+2}))/2 - C_{n/2} B_{n/2+1}$$
if $n > 2$ is even, where $C_n, B_n$ are the Catalan and Bell numbers, and $D_n$ is the number of partitions of $[n]$ up to reflection, which for $n=0,1,2,\dots$ is
$$ 1, 1, 2, 4, 11, 32, 117, \dots$$
(\url{https://oeis.org/A103293}).  A proof of this claim can be found in the ETP blueprint.  In particular, there are $4694$ equations of order at most $4$.

Below we record some specific equations appearing in this paper, using the alphabet $\x,\y,\z,\w,\uu,\vv$ in place of $0,1,2,3,4,5,\dots$ for readability.
\begin{align}
    \x &\formaleq \x & \hbox{(Trivial law)} \label{eq1}\tag{E1} \\
    \x &\formaleq \y & \hbox{(Singleton law)} \label{eq2}\tag{E2} \\
    \x &\formaleq \x \op \x & \hbox{(Idempotent law)} \label{eq3}\tag{E3} \\
    \x &\formaleq \x \op \y & \hbox{(Left-absorptive law)} \label{eq4}\tag{E4} \\
    \x &\formaleq \y \op \x & \hbox{(Right-absorptive law)} \label{eq5}\tag{E5} \\
    \x &\formaleq \x \op (\y \op \x) \label{eq10}\tag{E10} \\
    \x &\formaleq (\x \op \x) \op \x \label{eq23}\tag{E23} \\
    \x \op \x &\formaleq \y \op \z \label{eq41}\tag{E41} \\
    \x \op \y &\formaleq \y \op \x & \hbox{(Commutative law)} \label{eq43}\tag{E43} \\
    \x \op \y &\formaleq \z \op \w & \hbox{(Constant law)} \label{eq46}\tag{E46} \\
    \x &\formaleq \x \op (\x \op (\x \op \x)) \label{eq47}\tag{E47} \\
    \x &\formaleq \y \op (\y \op (\x \op \y))  \label{eq73}\tag{E73} \\
    \x &\formaleq (\x \op \x) \op (\x \op \x) \label{eq151}\tag{E151} \\
    \x &\formaleq (\y \op \x) \op (\x \op \z) & \hbox{(Central groupoid law)} \label{eq168}\tag{E168} \\
    \x &\formaleq (\x \op (\x \op \y)) \op \y \label{eq206}\tag{E206} \\
    \x &\formaleq ((\x \op \x) \op \x) \op \x \label{eq255}\tag{E255} \\
    \x \op \y &\formaleq \x \op (\y \op \z) \label{eq327}\tag{E327} \\
    \x &\formaleq (\x \op \y) \op \y \label{eq378}\tag{E378} \\
    \x \op \y &\formaleq (\z \op \x) \op \y \label{eq395}\tag{E395} \\
    \x &\formaleq \y \op ( (\z \op (\x \op (\y \op \z)))) & \hbox{(Tarski's axiom)} \label{eq543}\tag{E543} \\
    \x &\formaleq \y \op (\x \op ((\y \op \x) \op \y)) \label{eq677}\tag{E677} \\
    \x &\formaleq \x \op ((\x \op \x) \op (\x \op \x)) \label{eq817}\tag{E817} \\
    \x &\formaleq \x \op ((\y \op \z) \op (\x \op \z)) \label{eq854}\tag{E854}\\
    \x &\formaleq \y \op ((\y \op (\x \op \x)) \op \y) \label{eq1110}\tag{E1110} \\
    \x &\formaleq \y \op ((\y \op (\x \op \z)) \op \z) \label{eq1117}\tag{E1117} \\
    \x &\formaleq \y \op (((\x \op \y) \op \x) \op \y) \label{eq1286}\tag{E1286} \\
    \x &\formaleq (\y \op \x) \op (\x \op (\z \op \y)) & \hbox{(Weak central groupoids)}\label{eq1485}\tag{E1485} \\
    \x &\formaleq (\y \op \z) \op (\y \op (\x \op \z)) & \hbox{(Exp. $2$ abelian groups)} \label{eq1571}\tag{E1571} \\
    \x &\formaleq (\x \op \x) \op ((\x \op \x) \op \x) \label{eq1629}\tag{E1629} \\
    \x &\formaleq (\x \op \y) \op ((\x \op \y) \op \y) \label{eq1648}\tag{E1648} \\
    \x &\formaleq (\x \op \y) \op ((\y \op \y) \op \z) \label{eq1659}\tag{E1659} \\
    \x &\formaleq (\y \op \x) \op ((\x \op \z) \op \z) \label{eq1689}\tag{E1689} \\
    \x &\formaleq (\y \op \y) \op ((\y \op \x) \op \y) \label{eq1729}\tag{E1729} \\
    \x &\formaleq (\y \op (\x \op (\y \op x))) \op \y \label{eq2301}\tag{E2301}
\end{align}
    \begin{align}
        \x &\formaleq (\x \op ((\x \op \x) \op \x)) \op \x \label{eq2441}\tag{E2441} \\
        \x &\formaleq ((\y \op (\x \op \y)) \op \x) \op \y \label{eq2910}\tag{E2910} \\
        \x \op \y &\formaleq \x \op (\y \op (\x \op \y)) \label{eq3316}\tag{E3316} \\
        \x \op (\y \op \x) &\formaleq \x \op (\y \op \z) \label{eq4315}\tag{E4315} \\
        \x \op (\x \op \x) &\formaleq (\x \op \x) \op \x \label{eq4380}\tag{E4380} \\
        \x \op (\y \op \y) &= (\y \op \y) \op \x \label{eq4482}\tag{E4482} \\
        (\x \op \y) \op \z &\formaleq \x \op (\y \op \z) &\hbox{(Associative law)} \label{eq4512}\tag{E4512}\\
        \x \op (\y \op \z) &\formaleq (\y \op \z) \op \x \label{eq4531}\tag{E4531} \\
        \x &\formaleq \y \op ((\y \op (\y \op \x)) \op (\z \op \y)) &  \label{eq5093}\tag{E5093} \\
        \x &\formaleq (\y \op ((\x \op \y) \op \y)) \op (\x \op (\z \op \y)) & \hbox{(Sheffer stroke)} \label{eq345169}\tag{E345169} \\
        \x &\formaleq \y \op ((((\y \op \y) \op \x) \op \z) & \hbox{(Division in groups)} \label{eq42323216}\tag{E42323216} \\
        & \quad  \op (((\y \op \y) \op \y) \op \z)) \nonumber
\end{align}

\section{Author Contributions}

In a \href{https://github.com/teorth/equational_theories/blob/main/paper/contributions.md}{companion document} to this paper, the contributions of each author of this paper to the ETP are described, following the standard CRediT categories\footnote{\url{https://credit.niso.org/}}.  Below are the affiliations and grant acknowledgments of individual participants.  \TODO{Complete the affiliations here.  For contributors with no affiliations to report, enter `Unaffiliated'.}


\begin{itemize}
    \item Matthew Bolan: University of Toronto, matthew.bolan@mail.utoronto.ca
    \item Joachim Breitner: ...
    \item Jose Brox: IMUVA-Mathematics Research Institute, Universidad de Valladolid, josebrox@uva.es. Supported by a postdoctoral fellowship “Convocatoria 2021” funded by Universidad de Valladolid, and partially supported by grant PID2022-137283NB-C22 funded by MCIN/AEI/10.13039/501100011033 and ERDF “A way of making Europe”
    \item Mario Carneiro: ...
    \item Martin Dvorak: Institute of Science and Technology Austria, martin.dvorak@matfyz.cz
    \item Andr\'es Goens: University of Amsterdam, a.goens@uva.nl
    \item Aaron Hill: ...
    \item Harald Husum: harald.husum@gmail.com
    \item Zoltan A. Kocsis: University of New South Wales, z.kocsis@unsw.edu.au
    \item Bruno Le Floch: CNRS and Laboratoire de Physique Th\'eorique et Hautes \'Energies, Sorbonne Universit\'e, blefloch@lpthe.jussieu.fr
    \item Lorenzo Luccioli: University of Bologna, lorenzo.luccioli2@unibo.it
    \item Douglas McNeil: dsm054@gmail.com
    \item Alex Meiburg: ...
    \item Pietro Monticone: University of Trento, pietro.monticone@studenti.unitn.it
    \item Pace Nielsen: Department of Mathematics, Brigham Young University, pace@math.byu.edu
    \item Giovanni Paolini: University of Bologna, g.paolini@unibo.it
    \item Marco Petracci: University of Bologna, marco.petracci@studio.unibo.it
    \item Bernhard Reinke: Aix-Marseille Université, bernhard.reinke@univ-amu.fr
    \item David Renshaw: ...
    \item Marcus Rossel: Barkhausen Institut, marcus.rossel@barkhauseninstitut.org
    \item Cody Roux: Amazon Web Services, cody.roux@gmail.com
    \item J\'er\'emy Scanvic, Laboratoire de Physique, École Normale Supérieure de Lyon, jeremy.scanvic@ens-lyon.fr
    \item Shreyas Srinivas: ...
    \item Anand Rao Tadipatri: ...
    \item Terence Tao: Department of Mathematics, UCLA, tao@math.ucla.edu
    \item Vlad Tsyrklevich: vlad@tsyrklevi.ch
    \item Daniel Weber: ...
    \item Fan Zheng: ...

\end{itemize}


\bibliographystyle{plain}
\bibliography{references}

%delete this later
\listoftodos{}
\end{document}
