\chapter{Equation 1729}\label{1729-chapter}

In this chapter we study magmas that obey equation 1729,
\begin{equation}\label{1729}
  x = (y \op y) \op ((y \op x) \op y).
\end{equation}
for all $x,y$.  Using the squaring operator $Sy := y \op y$ and the left and right multiplication operators $L_y x := y \op x$ and $R_y x = x \op y$, this law can be written as
$$ L_{Sy} R_y L_{y} x = x.$$
This implies that $L_y$ is injective and $L_{Sy}$ is surjective, hence $L_{Sy}$ is invertible.  If $y$ is a square (i.e., $y \in SM$), then $L_y$ and $L_{Sy}$ are both invertible, hence now $R_y$ is also invertible, with inverse $R_y^{-1} = L_y L_{Sy}$.  We rewrite this as
\begin{equation}\label{lys}
  L_y = R_y^{-1} L_{Sy}^{-1}
\end{equation}
for all $y \in SM$.

We have the following procedure for extending a small magma $SM$ obeying \Cref{1729} to a larger one $M$:

\begin{theorem}[Extending a 1729 magma]\label{mag}  Let $SM$ be a magma obeying 1729, and let $N$ be another set disjoint from $SM$, and set $M := SM \uplus N$.  Suppose that we have a squaring map $S': N \to SM$ (which will complement the existing squaring map $S: SM \to SM$), and bijections $L'_a, R'_a: N \to N$ for all $a \in SM$ (which will complement the existing bijections $L_a, R_a: SM \to SM$ coming from $SM$), obeying the following axioms:
  \begin{itemize}
  \item (i) For all $a \in SM$, we have $L'_a = (R'_a)^{-1} (L'_{Sa})^{-1}$.
  \item (ii) For all $y \in N$, the elements $R'_a y \in N$ are distinct from each other and from $y$ as $a \in SM$ varies.
  \item (iii)  If $R'_a x = y$ for some $a \in SM$ and some $x,y \in N$, then $L'_{S'y} L_{R_{S'x}^{-1} a} y = x$.
  \item (iv)  For all $x \in N$, we have $(L'_{S'x})^2 x = x$.
  \end{itemize}
Suppose also that we have an operation $\op': N \times N \to M$ obeying the following axioms:
\begin{itemize}
  \item (v)  For all $x \in N$, we have $x \op' x = S'x$.
  \item (vi)  For all $y \in N$ and $a \in SM$, we have $R'_a y \op' y = L_{S'y}^{-1} a$.
  \item (vii)  For all $x,y \in N$ with $x \op' y$ not already covered by rules (v) or (vi), we have $x \op' y = z$ for some $z \in N$.  Furthermore, $z \op' x = (L'_{S'x})^{-1} y$.
\end{itemize}
Then one can endow $M$ with an operation $\op'': M \times M \to M$ obeying 1729 defined as follows:
\begin{itemize}
\item  If $a,b \in SM$, then $a \op'' b = a \op b$.
\item  If $a \in SM$ and $x \in N$, then $a \op'' b := L'_a b$.
\item  If $x \in N$ and $a \in SM$, then $b \op'' a := R'_a b$.
\item  If $x,y \in N$, then $x \op'' y := x \op' y$.
\end{itemize}
Furthermore, the 817 law $x \op'' SS' x = x$ fails for any $x \in N$.
\end{theorem}

\begin{proof}  We need to show that $\op''$ verifies the law \Cref{1729}.  In the case when $x,y \in SM$, then the claim follows from the fact that $SM$ already obeyed this equation.  If $x$ was equal to an element $a \in SM$ and $y \in N$, then by construction the law is equivalent to $L'_{Sa} R'_a L'_a y = y$, which follows from axiom (i).

Now suppose that $x \in N$ and $y$ is equal to some element $a$ of $SM$.  From axiom (v) we have $x \op'' x = S'x$, and then this case of \Cref{1729} becomes
$$L_{S'x} ( R'_a x \op' x ) = a$$
which follows from axiom (vi).  So the only remaining case is when $x,y \in N$.  Using axiom (ii), we can divide into cases:
\begin{itemize}
\item Case 1: $x=y$.  Then by (v) we need to show that $L'_{S'x} L'_{S'x} x = x$, which follows from axiom (iv).
\item Case 2: $y = R'_a x$ for some $a \in SM$.  Then by axiom (vi), we need to show that $L'_{S'y} L'_{L_{S'x}^{-1} a} y = x$, which follows from axiom (iii).
\item Case 3: We are not in case 1 or case 2.  Then by axiom (vii), we have $y \op'' x = z$ for some $z \in N$ with $z \op'' y = (L'_{S'y})^{-1} x$.  But this implies $L'_{S'y} (z \op'' y) = x$, which is \Cref{1729}.
\end{itemize}

We have now verified that $\op''$ obeys 1729.  For any $x \in N$, we have $x \op'' SS' x = R'_{SS' x} x$, and so the final claim follows from axiom (ii).
\end{proof}

To build a magma obeying 1729 but not 817, it thus suffices to produce
\begin{itemize}
  \item a 1729 magma $SM$;
  \item a set $N$ of ``non-squares'';
  \item a squaring map $S': N \to SM$;
  \item bijections $L'_a, R'_a: N \to N$ for all $a \in SM$ obeying the axioms (i)-(iv); and
  \item an operation $\op': N \times N \to SM \uplus N$ obeying the axioms (v)-(vii).
\end{itemize}

The magma $SM$ is defined as follows:

\begin{definition}[Definition of $SM$]\label{sm-def}\lean{Eq1729.SM}\leanok Take $SM$ to be a countably infinite abelian group of exponent $4$, generated by generators $E_n$ for $n \in \N$ subject to the relations $4E_n=0$.
\end{definition}

\begin{lemma}[Basic properties of $SM$]\label{sm-1729}\lean{Eq1729.SM_square_eq_double, Eq1729.SM_square_square_eq_zero, Eq1729.SM_obeys_1729}\leanok $SM$ is a 1729 magma, the squaring operation $S: SM \to SM$ is just the doubling map $Sa = 2a$, and the double squaring map $S^2: SM \to SM$ is constant: $S^2 a = 0$ for all $a \in SM$.
\end{lemma}

\begin{proof}\leanok Routine verification.
\end{proof}

We now define $N$, as well as some Cayley graph structures on it.

\begin{definition}[Definition of $N$]\label{n-def}\lean{Eq1729.N, Eq1729.N_order, Eq1729.parent}\leanok Take $N$ to be the free non-abelian group with a generator $e_a$ for each $a \in SM$, thus $N$ is the set of reduced words using the alphabet $e_a$, $e_a^{-1}$.  Two elements $x,y \in SM$ are said to be \emph{adjacent} if $x = e_a y$ or $y = e_a x$ for some $a \in SM$; this defines a left Cayley graph on $N$.  We make a partial ordering $\leq$ on $N$ by declaring $y \leq x$ if $y$ is a right subword of $x$ (or equivalently, $y$ is on the unique simple path from $1$ to $x$).  For instance, if $a,b,c \in SM$ are distinct, then
  $$ 1 \leq e_c \leq e_b^{-1} e_c \leq e_a e_b^{-1} e_c.$$
If $x \in N$ is not the identity, we define the \emph{parent} of $x$ to be the unique element $y \in N$ adjacent to $x$ whose reduced word is shorter.  For instance, the parent of $e_a e_b^{-1} e_c$ is $e_b^{-1} e_c$.
\end{definition}

\begin{lemma}[Basic properties of $N$]\label{n-prop}\uses{n-def}\lean{Eq1729.N_countable, Eq1729.N_order}\leanok  $N$ is countable, and $\leq$ is a partial ordering.
\end{lemma}

\begin{proof}\leanok Routine verification.
\end{proof}

We will define the right multiplication operators $R'_a: N \to N$ using the group action:

\begin{definition}[Definition of $R'_a$]\label{ra-defn}\uses{sm-def, n-def}\lean{Eq1729.R'}\leanok We set
  \begin{equation}\label{ra-def}
    R'_a x := x e_a
  \end{equation}
  for all $a \in SM$ and $x \in N$.
\end{definition}

\begin{lemma}[Basic properties of $R'_a$]\label{ra-prop}\uses{ra-defn}\lean{Eq1729.R'_axiom_iia, Eq1729.R'_axiom_iib}\leanok  The operators $R'_a$ are bijective and obey axiom (ii).
\end{lemma}

\begin{proof}\leanok Routine verification.
\end{proof}

We defer construction of the squaring map $S': N \to SM$ for now, but turn to left-multiplication. From two applications of \Cref{lys} and the exponent 4 hypothesis we have
$$ L'_a = (R'_a)^{-1} (L'_{2a})^{-1} = (R'_a)^{-1} L'_0 R'_{2a}.$$
Thus, once $L'_0$ is specified, we can \emph{define} $L'_a$ for all other $a \in SM$ by the rule
\begin{equation}\label{la0}
  L'_a := (R'_a)^{-1} L'_0 R'_{2a}.
\end{equation}
Furthermore, from the $a=0$ case of \Cref{lys} we must also have the axiom
\begin{itemize}
  \item (i') $(L'_0)^2 = (R'_0)^{-1}$.
\end{itemize}

Conversely, we have

\begin{lemma}[Using $L'_0$ to construct $L'_a$]\label{l0-la}\lean{Eq1729.L',Eq1729.L'_0_eq_L₀'}\leanok  Suppose we have a bijection $L'_0: N \to N$ that obeys axiom (i'), and then define $L'_a$ for all $a \in SM$ by the formula \eqref{la0}.  Then this recovers $L'_0$ when $a=0$ (to formalize this it may be convenient to give $L'_0$ and $L'_a$ distinct names), and the $L'_a$ are all bijections and obey axiom (i).  Furthermore, we have
\begin{equation}\label{la1}
  (L'_a)^{-1} := (R'_{2a})^{-1} L'_0 R'_0 R'_{a}
\end{equation}
for all $a \in SM$.
\end{lemma}

\begin{proof}\leanok Routine verification.
\end{proof}

We now write the other remaining axioms in terms of $L'_0$ rather than $L'_a$ using \Cref{la0}, \Cref{la1}, \Cref{ra-def}, and the magma law on $SM$:
\begin{itemize}
\item (iii')  If $R'_a x = y$ for some $a \in SM$ and some $x,y \in N$, then $(R'_{S'y})^{-1} L'_0 R'_{2S'y} (R'_{a - S'x})^{-1} L'_0 e_{2(a - S'x)} y = x$.
\item (iv')  For all $x \in N$, we have $(R'_{S'x})^{-1} L'_0 R'_{2S'x} (R'_{S'x})^{-1} L'_0 R'_{2S'x} x = x$.
\item (v)  For all $x \in N$, we have $x \op' x = S'x$.
\item (vi')  For all $y \in N$ and $a \in SM$, we have $e_a y \op' y = a - S'y$.
\item (vii')  For all $x,y \in N$ with $x \op' y$ not already covered by rules (v) or (v'), we have $x \op' y = z$ for some $z \in N$.  Furthermore, $z \op' x = (R'_{2S'x})^{-1} L'_0 e_0 e_{S'x} y$.
\end{itemize}

\begin{lemma}[Reduction to new axioms]\label{axiom-reduce}\label{Eq1729.reduce_to_new_axioms}\leanok  Suppose we can find a function $S': N \to SM$, a bijection $L'_0: N \to N$, and an operation $\op': N \times N \to SM \uplus N$ obeying axioms (i'), (iii'), (iv'), (v), (vi'), (vii').  Then there exists a magma obeying 1729 but not 817.
\end{lemma}

\begin{proof}\uses{l0-la, ra-prop, mag}\leanok  Construct the $L'_a$ using \Cref{l0-la}.  By \Cref{ra-prop} and direct verification we can noe verify axioms (i)-(vii), and then the claim follows from \Cref{mag}.
\end{proof}

Our task is now to find a function $S': N \to SM$, a bijection $L'_0: N \to N$, and an operation $\op': N \times N \to M$ obeying axioms (i'), (iii'), (iv'), (v), (vi'), (vii').

We will again use a greedy construction for this, but with some modifications.  Firstly, the axiom (i'), together with \eqref{ra-def} means that we cannot restrict $L'_0$ to be partially defined on just finitely many values: any relation of the form
$$ L'_0 x = y$$
for some $x,y \in N$ would automatically imply that
\begin{equation}\label{itero}
 L'_0 (R'_0)^n x = (R'_0)^n y
\end{equation}
and also
\begin{equation}\label{itero-2}
  L'_0 (R'_0)^n y = (R'_0)^{n-1} x
\end{equation}
for all $n \in \Z$.  Thus, $L'_0$ becomes defined on two right cosets $\langle e_0 \rangle x$, $\langle e_0 \rangle y$ of $N$, where $\langle e_0 \rangle := \{ e_0^n: n \in \Z\}$ is an infinite cyclic subgroup of $N$.  In general, we will require that $L'_0$ is defined on a finite union of cosets of $\langle e_0\rangle$.

In a somewhat similar vein, axiom (vii'), if iterated naively, would mean that a given entry $x \op' y = z$ of the multiplication table could potentially generate an infinite sequence of further entries, which unfortunately do not have as regular a pattern as the iterations \Cref{itero}, \Cref{itero-2} of axiom (i').  So we will need to truncate this iteration by creating an addition category of ``pending'' identities $I[x,y,z]$ of the form ``$z \op' x = (R'_{2S'x})^{-1} L'_0 R'_0 e_{S'x} y$'' for some $x,y,z \in N$, which will be temporarily undefined because $S'x$ is undefined. More precisely,

\begin{definition}[Partial solution]\label{part-sol}\uses{sm-def, n-def}\lean{Eq1729.PartialSolution}\leanok  A \emph{partial solution} $(L'_0, \op', S', {\mathcal I})$ is a collection of the following data:
\begin{itemize}
  \item A partially defined function $L'_0: N \to N$, defined on a finite union of right cosets of $\langle e_0\rangle$;
  \item A partially defined operation $\op': N \times N \to M$, defined on a finite set;
  \item A partially defined function $S': N \to SM$, defined on a finite set; and
  \item A finite collection ${\mathcal  I}$ of ``pending identities'' $I[x,y,z]$, which one can think of either as ordered triples of elements $x,y,z \in N$, or as formal strings of the form ``$z \op' x = (R'_{2S'x})^{-1} L'_0 R'_0 R'_{S'x} y$'' for some $x,y,z \in N$.
\end{itemize}

Furthermore, the following axioms are satisfied:
\begin{itemize}
  \item (i'')  $L'_0 x$ is defined and equal to $y$, then we have the identities \Cref{itero}, \Cref{itero-2} for all $n\in \Z$.
  \item (S) If $S'x$ is defined for some $x \in N$, then $S'y$ is defined for all $y \leq x$.
  \item (iii'') If $R'_a x = y$ for some $a \in SM$ and some $x,y \in N$, and $S'x, S'y$ are defined, then $(R'_{S'y})^{-1} L'_0 R'_{2S'y} (R'_{a-S'x})^{-1} L'_0 R'_{2(a-S'x)} y$ is defined and equal to $x$.
  \item (iv'') If $x \in N$ is such that $S'x$ is defined, then $(R'_{S'x})^{-1} L'_0 R'_{2S'x} (R'_{S'x})^{-1} L'_0 R'_{2S'x} x$ is defined and equal to $x$.
  \item (v'')  If $x \in N$ and $x \op' x$ is defined, then $S'x$ is defined and equal to $x \op' x$.
  \item (vi'')  For all $y \in N$ and $a \in SM$, if $e_a y \op' y$ is defined, then $a - S'y$ is defined and equal to $R'_a y \op' y$.
  \item (vii'')  For all $x,y \in N$ and $x$ is not equal to $y$ or $R'_a y$ for any $a \in SM$, and $x \op' y$ is defined, then it is equal to some $z \in N$.  Furthermore, either $I[x,y,z]$ is a pending identity, or else $z \op' x$ and $(R'_{2S'x})^{-1} L'_0 R'_0 R'_{S'x} y$ are defined and equal to each other.
  \item (P) If $I[x,y,z]$ is a pending identity, then $x,y,z \in N$, and $Sx$ and $z \op' x$ are undefined.  Furthermore, $z$ is not equal to $x$ or $R'_a x$ for any $a \in SM$.
\end{itemize}

We say that one partial solution $(\tilde L'_0, \tilde \op', \tilde S', \tilde {\mathcal I})$ \emph{extends} another if $(L'_0, \op', S',  {\mathcal I})$ if $\tilde L'$ is an extension of $L'_0$, $\tilde \op'$ is an extension of $\op'$, and $\tilde S'$ is an extension of $S'$.  (No constraint is imposed on the final components $\tilde {\mathcal I}, {\mathcal I}$.)  This is a preordering.
\end{definition}

\begin{lemma}[Existence of partial solution]\label{partial-exist}\uses{part-sol}\lean{Eq1729.TrivialPartialSolution}\leanok  There exists a partial solution.
\end{lemma}

\begin{proof}\leanok Set $L'_0, \op', S'$ to be empty functions, and have the set of pending identities to also be empty.  The verification of the required axioms is then routine.
\end{proof}

\begin{lemma}[Chain of partial solutions]\label{chain}\uses{part-sol}\lean{Eq1729.use_chain}\leanok  Suppose that one has a sequence
$(L'_{0,n}, \op'_{n}, S'_{n}, {\mathcal I}_n)$ of partial solutions, each one an extension of the previous, such that for any $x, y \in N$, $L'_{0,n} x$, $x \op'_{n} y$, and $S'_n x$ are defined for some $n$.  Then there exists a 1729 magma that does not obey 817.
\end{lemma}

\begin{proof}\uses{axiom-reduce}\leanok  Take the direct limit of the chain to obtain total functions $L'_0, \op', S'$.  The axioms (i'), (iii'), (iv''), (v''), (vi''), (vii'') of the partial solutions then easily imply that the direct limit obeys the axioms (i'), (iii'), (iv'), (v), (vi'), (vii') (one also uses axiom (P) to note that all pending identities disappear in the direct limit). The claim now follows from \Cref{axiom-reduce}.
\end{proof}


Now we seek to enlarge a partial solution. We first make an easy observation:

\begin{proposition}[Enlarging $L'_0$]\label{enlarge-l0}\lean{Eq1729.enlarge_L₀'}\leanok  Suppose one has a partial solution in which $L'_0 x$ is undefined for some $x \in N$.  Then one can extend the partial solution so that $L'_0 x$ is now defined.
\end{proposition}

\begin{proof} \leanok By axiom (i''), $L'_0 (R'_0)^n x$ is undefined for every integer $n$.  Let $d = E_m$ be a generator of $SM$ that does not appear as a component of any index of any of the generators $e_a$ appearing anywhere in the partial solution; such a $d$ exists due to the finiteness hypotheses.  We set $L'_0 x := e_d$, and then extend by \Cref{itero}, \Cref{itero-2}, thus
$$L'_0 (R'_0)^n x := (R'_0)^n e_d$$
and
$$L'_0 (R'_0)^n e_d := (R'_0)^{n-1} x.$$
Because of the new nature of $d$, no collisions in the partial function $L_0$ are created by this operation.  It is then easy to check that axiom (i'') is preserved by this operation, whereas none of the other axioms (S), (iii''), (iv''), (v''), (vi''), (vii''), (P) are affected by this extension.
\end{proof}

Next, we provide a tool for enlarging the domain of definition of $S'$:

\begin{proposition}[Enlarging $S'$]\label{enlarge-S}\lean{Eq1729.enlarge_S'}\leanok  Suppose one has a partial solution in which $S'x$ is undefined for some $x \in N$.  Then one can extend the partial solution so that $S'x$ is now defined.
\end{proposition}

\begin{proof} By iteration we may assume inductively that $S'y$ is already defined for all $y$ on the unique path in the Cayley graph from $1$ to $x$.  (This hypothesis is vacuous if $x=1$.)  Let $y_0$ be the parent of $x$, that is to say the unique neighbor of $x$ in the path to $1$ (this is only defined for $x \neq 1$), then by axiom (i'') $y_0$ is the unique neighbor for which $S'y_0$ is defined, and we either have $x = R'_a y_0$ or $R'_a x = y_0$ for some $a \in SM$.

Let $d_0, d_1, d_2 \in SM$ be distinct generators $E_{n_0}, E_{n_1}, E_{n_2}$ of $SM$ that do not appear in the index $a$ of any $e_a$ that currently appears in the partial solution (of which there are only finitely many).  We also set some further distinct generators $d'_{y,z}, d''_{y,z} \in SM$ for $y,z \in SM$ that are distinct from each other and all previous generators (this is possible as we have infinitely many generators).  We set $S'x := e_{d_0}$; this preserves axiom (S).  In order to retain axiom (iv''), we now also set
$$ L'_0 R'_{2d_0} x := e_{d_1}$$
$$ L'_0 R'_{2e_{d_0}} (R'_{S'x})^{-1} e_{d_1} := e_{d_0} x$$
and then extend these choices using \Cref{itero}, \Cref{itero-2} to recover axiom (i'), thus for instance
$$ L'_0 (R'_0)^n e_{2d_0} x := (R'_0)^n e_{d_1}$$
and
$$ L'_0 (R'_0)^n e_{d_1} x := (R'_0)^{n-1} e_{2d_0}$$
for all $n \in \Z$.  To retain axiom (iii''), we perform the following actions:
\begin{itemize}
\item If $y_0$ is undefined, do nothing.
\item If $e_a x = y_0$ for some $a \in SM$, set $L'_0 R'_{2(a-d_0)} y_0 := e_{d_2}$ and $L_0 R'_{2S'y_0} (R'_{a-d_0})^{-1} e_{d_2} := e_{S'y_0} x$.  Then extend using \Cref{itero}, \Cref{itero-2}.
\item If $x = e_a y_0$ for some $a \in SM$, and $L'_0 (R'_{2(a-Sy_0)})^{-1} x$ is already defined, set $L'_0 R'_{2d_0} (R'_{a-S'y_0})^{-1} L'_0 R'_{2(a-Sy_0)} x :=R'_{d_0}x$, and extend using \Cref{itero}, \Cref{itero-2}.
\item If $x = e_a y_0$ for some $a \in SM$, and $L'_0 (R'_{2(a-Sy_0)})^{-1} x$ is not yet defined, set $L'_0 R'_{2(a-Sy_0)} x := e_{d_2}$ and $L'_0 R'_{2d_0} (R'_{a-Sy_0})^{-1} e_{d_2} := e_{d_0} x$, and extend using \Cref{itero}, \Cref{itero-2}.
\end{itemize}
To retain axiom (P), we perform the following actions for each pending identity of the form $I[x,y,z]$ for some $y,z$, executed in some arbitrary order.
\begin{itemize}
\item Remove this identity $I[x,y,z]$ from the list of pending identities.
\item Set $L_0 R'_0 e_{d_0} y := e_{d'_{y,z}}$, and $x' \op y' := z'$, where $(x',y',z') := (z, x, (R'_{2d_0})^{-1} e_{d'_{y,z}})$.
\item If $S'x'$ is undefined, add $I[x',y',z']$ as a pending identity.
\item If instead $S'x'$ is defined, but $L'_0 R'_0 R'_{S'x'} y'$ is undefined, set $L'_0 R'_0 R'_{S'x'} y' = e_{d''_{y,z}}$.  Then set $x'' \op y'' = z''$, where $(x'',y'',z'') := (z', x', (R'_{2S'x'})^{-1} e_{d''_{y,z}})$.  Then add $I[x'',y'',z'']$ as a pending identity.
\item If instead $S' x$ and $L_0 R'_0 R'_{S'x'} y'$ are both defined, set $x'' \op y'' = z''$, where $(x'',y'',z'') := (z', x', (R'_{2S'x'})^{-1} L_0 R'_0 R'_{Sx'} y')$.  Then add $I[x'',y'',z'']$ as a pending identity.
\item Extend all the new definitions of $L'_0$ using \Cref{itero}, \Cref{itero-2}.
\end{itemize}
One can check that these definitions do not cause any collisions in the partial function $L'_0$, and that axioms (i'), (iii'), (iv''), (vii'), (P) are preserved; the remaining axioms (v''), (vi') are unaffected by this extension.
\end{proof}

Finally, we give a tool for enlarging $\op$:

\begin{proposition}[Enlarging $\op$]\label{enlarge-op}\lean{Eq1729.enlarge_op}\leanok  Suppose one has a partial solution in which $x \op' y$ is undefined for some $x,y \in N$.  Then one can extend the partial solution so that $x \op' y$ is now defined.
\end{proposition}

\begin{proof}\uses{enlarge-S}  By applying \Cref{enlarge-S} as needed, we may assume without loss of generality that $S'x$ and $S'y$ are already defined (among other things, this removes the possibility that $x \op' y$ is part of a pending identity).  If this makes $x \op' y$ defined, then we are done, so we may assume that this is not the case.

We now divide into cases:

Case 1: $x=y$.  In this case we set $x \op' y := S'x$.  It is clear that this preserves axiom (v''), and no other axiom is impacted.

Case 2: If $x = R'_a y$ for some $a \in SM$, then we set $x \op' y := a - S'y$.  This preserves axiom (vi''), and no other axiom is impacted.

Case 3: If $x$ is not equal to $y$ or $R'_a y$ for any $a \in SM$.  Let $d_0, d_1 \in SM$ be a generator that does not appear as a component of any index of any of the generators $e_a$ appearing anywhere in the partial solution.  We set $x \op' y := z$ with $z := e_{d_0}$.

This temporarily disrupts axiom (vii'').  To recover it, we perform the following actions.
\begin{itemize}
  \item If $L'_0 R'_0 R'_{S'x} y$ is not currently defined, we set it equal to $e_{d_1}$, and extend by \Cref{itero}, \Cref{itero-2}.
  \item Set $x' \op y' = z'$, where $(x',y',z') := (z, x, (R'_{2S'x})^{-1} L_0 R'_0 e_{Sx} y)$.
  \item Add $I[x',y',z']$`as a pending identity.
\end{itemize}

One can check that these definitions do not cause any collisions in the partial function $L'_0$, and that axioms (i'), (vii'), (P) are preserved; the remaining axioms (S), (iii''), (iv''), (v''), (vi'') are unaffected by this extension.
\end{proof}


\begin{theorem}[1729 does not imply 817]\label{1729_refute_817}\lean{Eq1729.not_817}\leanok There exists a magma that obeys equation 1729 but not equation 817.
\end{theorem}

\begin{proof}\uses{partial-exist, enlarge-l0, enlarge-S, enlarge-op,chain}
  Starting from \Cref{partial-exist} and applying \Cref{enlarge-l0}, \Cref{enlarge-S}, \Cref{enlarge-op} alternatingly, one can find a chain of partial solution that are total in the limit.  The claim now follows from \Cref{chain}.
\end{proof}
