\chapter{Weak central groupoids}\label{weak-central-groupoids-chapter}

In this chapter we study weak central groupoids \Cref{eq1485},
\begin{equation}\label{1485}
   x = (y \op x) \op (x \op (z \op y)).
\end{equation}

The first observation is that this law is equivalent to its dual:

\begin{lemma}[1485 equivalent to 2162]\label{1485-dual}\uses{eq1485}\leanok\lean{WeakCentralGroupoid.dual_eqn}
  \Cref{eq1485} is equivalent to the dual law
\begin{equation}\label{2162} x = ((y \op z) \op x) \op (x \op y)
\end{equation}
(equation 2162).
\end{lemma}

\begin{proof}  It suffices to prove that \eqref{1485} implies \eqref{2162}.  Write $w = y \op z$, then from \eqref{1485} we can write $z = z_1 \op z_2$ with $z_1 = z \op z$ and $z_2 = z \op (z \op z)$, and then by \eqref{1485}
  $$ y = (z_2 \op y) \op (y \op (z_1 \op z_2)) = (z_2 \op y) \op w.$$
From another application of \eqref{1485} we have
$$ x = (w \op x) \op (x \op ((z_2 \op y) \op w)) = ((y \op z) \op x) \op (x \op y)$$
as required.
\end{proof}

Given a weak central groupoid $G$, define a directed graph with vertices in $G$ by declaring $x \to y$ if and only if $y = x \op z$ for some $z$.  There is an equivalent characterization of this graph:

\begin{lemma}[Equivalent characterization of graph]\label{graph-dual}\uses{eq1485}\leanok\lean{WeakCentralGroupoid.Path.def'}
  One has $x \to y$ if and only if $x = w \op y$ for some $w$.
\end{lemma}

\begin{proof}\uses{1485-dual}  If $x \to y$ then $y = x \op z$, then writing $z = z_1 \op z_2$ as before we obtain
$$ x = (z_2 \op x) \op (x \op (z_1 \op z_2)) = (z_2 \op x) \op y$$
giving the forward implication.  The backwards implication follows by duality.
\end{proof}

Define a \emph{good path} in $G$ to be a path of the form $$x \to x \op y \to y$$ for some $x,y \in G$ (we allow loops).  By the above lemma, this is a path in $G$.  The following claims are clear from definition and the above lemma:
\begin{itemize}
\item[Claim 1:]  If $x,y \in G$ then there is exactly one good path $x \to z \to y$ from $x$ to $y$.
\item[Claim 2:] Any edge $x \to y$ in the directed graph is the initial segment of some good path $x \to y \to z$.
\item[Claim 3:] Any edge $x \to y$ in the directed graph is the final segment of some good path $w \to x \to y$.
\end{itemize}
Slightly more non-trivial is

\begin{lemma}[Claim 4]\label{claim-4}\uses{eq1485}\leanok\lean{WeakCentralGroupoid.isGood_five}
  If $a \to b \to c \to d \to e \to a$ is a 5-cycle in the directed graph, and $a \to b \to c$ and $c \to d \to e$ are good paths, then $b \to c \to d$ is also good.
\end{lemma}

\begin{proof}\uses{graph-dual}  If $a \to b \to c$ is good then $b = a \op c$; if $c \to d \to e$ is good then $d = c \op e$; and if $e \to a$ then $a = e \op z$ for some $z$ by definition.  By \eqref{2162} we then have
  $$ c = ((e \op z) \op c) \op (c \op e) = b \op d$$
  so $b \to c \to d$ is good.
\end{proof}

Conversely, we have

\begin{lemma}[Reversing the claims]\label{rev-claim}\uses{claim-4}\leanok\lean{RelaxedWeakCentralGroupoid.strictify}
  Let $G$ be a directed graph, with some paths of length two in the graph designated as ``good'', in such a way that Claims 1-4 hold.  Then there is a weak central groupoid structure on the vertices of $G$ such that the good paths are precisely the paths of the form $x \to x \op y \to y$.
\end{lemma}

\begin{proof}  Define an operation $\op: G \times G \to G$ by defining $x \op y$ to be the unique vertex $z$ for which one has a good path $x \to z \to y$; this is well-defined by Claim 1, and by Claims 2-3, the property $x \to y$ holds if and only if $y = x \op z$ for some $z$, and also if and only if $x = w \op y$ for some $w$. In particular, for all $x,y,z$, we have a $5$-cycle
$$ y \to y \op x \to x \to x \op (z \op y) \to (z \op y) \to y$$
with $y \to y \op x \to x$ and $x \to x \op (z \op y) \to (z \op y)$ good, hence by Claim 4 we have \eqref{1485} as required.
\end{proof}

This gives us a graph-theoretical route to construct weak central groupoids.  We first introduce a weaker version of Claim 1:

\begin{itemize}
\item[Claim 1':]  If $x,y \in G$ then there is at least one good path $x \to z \to y$ from $x$ to $y$.
\end{itemize}

Let us call a \emph{relaxed weak central groupoid} a directed graph with some paths of length 2 designated as ``good'' that obeys Claims 1', 2, 3, 4.

We also define a \emph{partial weak central groupoid} to be a directed graph with some paths of length 2 that obeys Claim 4 as well as the following opposite weakening of Claim 1:

\begin{itemize}
  \item[Claim 1'':]  If $x,y \in G$ then there is at most one good path $x \to z \to y$ from $x$ to $y$.
\end{itemize}

If we can upgrade Claim 1'' to Claim 1, and we also have Claim 2 and Claim 3, then we call this a \emph{complete weak central groupoid}, and by the previous proposition this is in correspondence with \Cref{1485}.

Let $G_0$ be a relaxed weak central groupoid.  A \emph{partial extension} of $G_0$ is a partial weak central groupoid $G$ with a ``projection map'' $\pi: G \to G_0$, which is a homomorphism in the sense that the image $\pi(x) \to \pi(y)$ of any edge $x \to y$ in $G$ is an edge in $G_0$, the image $\pi(x) \to \pi(y) \to \pi(z)$ of any good path $x \to y \to z$ in $G$ is a good path in $G_0$, and the image $\pi(x) \to \pi(y) \to \pi(z)$ of any bad path $x \to y \to z$ in $G$ is a bad path in $G_0$.  Note that Claim 4 for $G$ is then automatic from Claim 4 of the base $G_0$.  The extension is \emph{complete} if the partial weak central groupoid is complete.

We have the following convenient completion property:

\begin{proposition}[Completion property]\label{greedy-prop}\uses{eq1485}  Let $G_0$ be a directed graph obeying claims 1', 2, 3, 4.  Then any finite partial extension of $G_0$ with carrier $G_0 \times \N$ (and projection map $\pi(a,n) = a$) can be completed to a complete extension.
\end{proposition}

\begin{proof}  By the previous comments, we can ignore Claim 4 as it is automatic, and focus on completing the partial weak central groupoid on $G$ to a complete weak central groupoid by ensuring Claims 1, 2, 3 hold.  By the usual greedy algorithm, it suffices to show that any individual failure of Claim 1, 2 or 3 can be resolved by adding some finite number of edges to the graph.

  Suppose first that Claim 2 fails, that is to say the partial weak central groupoid contains an edge $(a,n) \to (b,m)$ that is not the initial segment of any good path.  Since the base relaxed weak central groupoid $G_0$ obeys Claim 2, we can find a good path $a \to b \to c$ in the base.  We then pick a natural number $l$ not previously occuring in the partial weak central groupoid, and adjoin the edge $(b,m) \to (c,l)$ to that partial weak central groupoid.  All new paths created in this way are declared good or bad depending on whether their images in $G_0$ are good or bad, in particular $(a,n) \to (b,m) \to (c,l)$ is good.  This can be checked to still be a partial extension of $G_0$ (no violation of Claim 1'' is created), and now Claim 2 is resolved at for the edge $(a,n) \to (b,m)$.  A similar argument permits one to resolve any violations of Claim 3.

If Claim 1 is violated, then there is a pair $(a,n), (b,m)$ that currently has no good path of length two in the partial weak central groupoid.  As the base relaxed weak central groupoid $G_0$ obeys Claim 1', we can find a good path $a \to c \to b$ in $G_0$.  We then pick a natural number $l$ not previously occuring, and adjoin the edges $(a,n) \to (c,l) \to (b,m)$. All new paths created in this way are declared good or bad depending on whether their images in $G_0$ are good or bad, in particular $(a,n) \to (c,l) \to (b,m) \to (c,l)$.  One can check that this is still a partial extension of $G_0$ (no violation of Claim 1'' is created), and now Claim 1 is resolved at the pair $(a,n), (b,m)$.
\end{proof}

\begin{theorem}[1485 does not imply 1483]\label{1485-refutes}\uses{eq1485}  \Cref{eq1485} does not imply any of the following laws:
\begin{itemize}
\item Equation 3457: $x \op x = x \op ((x \op x) \op y)$.
\item Equation 2087: $x = ((y \op x) \op x) \op (x \op x)$.
\item Equation 2124: $x = ((y \op y) \op x) \op (x \op x)$.
\item Equation 3511: $x \op y = x \op ((x \op y) \op x)$.
\end{itemize}
\end{theorem}

\begin{proof}\uses{greedy-prop, rev-claim}  Computer check reveals that the carrier $G_0=\{0,1,2,3 ,4\}$ with incidence matrix
$$
\begin{pmatrix}
  1 & 1 & 0 & 0 & 1 \\
  1 & 1 & 0 & 1 & 0 \\
  0 & 0 & 1 & 1 & 1 \\
  1 & 0 & 1 & 0 & 0 \\
  0 & 1 & 1 & 0 & 0 \\
  \end{pmatrix}
$$
is a relaxed weak central groupoid if we declare the paths $0 \to 0 \to 0$, $0 \to 0 \to 1$, $0 \to 1 \to 1$, $1 \to 0 \to 0$, $1 \to 1 \to 0$, $1 \to 1 \to 1$ to be bad, and all other paths in the directed graph to be good.  Another way of thinking about this relaxed weak central groupoid is via the relaxed multiplication table
$$
\begin{matrix}
1 & 4 & 4 & 1 & 0\\
3 & 0 & 3 & 1 & 0\\
3 & 4 & 3,4 & 2 & 2 \\
0 & 0 & 2 & 2 & 0,2 \\
1 & 1 & 2 & 1,2 & 2
\end{matrix}
$$
where some entries of the table have multiplicity due to multiple good paths.  For instance, because of the good paths $2 \to 3 \to 2$ and $2 \to 4 \to 2$, $2 \op 2$ is ``permitted'' to be either $3$ or $4$.

Observe that the directed graph consisting of a single bidirectional edge $(2,0) \leftrightarrow (4,0)$ (i.e., the directed edges $(2,0) \to (4,0)$ and $(4,0) \to (2,0)$) is a partial extension of $G$.  Applying \Cref{greedy-prop}, we can complete this to a complete weak central groupoid $G$ with carrier $\{0,1,2,3,4\} \times \N$, such that $(2,0) \op (2,0) = (4,0)$.  The first coordinates of this law will act in a fashion consistent with the above table.  For instance, $(0,n) \op (1,m)$ will lie in $\{4\} \times \N$ for any $n,m$, while $(2,n) \op (2,m)$ will lie in $\{3,4\} \times \N$.  Now we can check the violations:
\begin{itemize}
\item Violation of 3457:  Take $x = (1,0)$, $y = (4,0)$.  Then $x \op x$ must be in $\{0\} \times \N$, but $x \op ((x \op x) \op y)$ can be checked to be in $\{3\} \times \N$.
\item Violation of 2087:  Take $x = (1,0)$, $y = (2,0)$.  Then $x \op y$ must be in $\{1\} \times \N$, but $((y \op x) \op x) \op (x \op x)$ can be checked to be in $\{3\} \times \N$.
\item Violation of 2124:  Take $x = (1,0)$, $y = (2,0)$.  Then $x \op y$ must be in $\{1\} \times \N$, but $((y \op y) \op x) \op (x \op x)$ can be checked to be in $\{3\} \times \N$.
\item Violation of 3511:  Take $x = (1,0)$, $y = (3,0)$.  Then $x \op y$ must be in $\{1\} \times \N$, but $x \op ((x \op y) \op x)$ can be checked to be in $\{3\} \times \N$.
\end{itemize}
\end{proof}
