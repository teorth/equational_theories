\chapter{Implications between selected laws}\label{implications-chapter}

We collect here some notable implications between the the selected laws in \Cref{subgraph-eq}.   By \Cref{sound-complete}, every implication can basically be established by a finite number of rewrites.  In most cases, the sequence of rewrites is quite straightforward, and the implication is very easy, but we record some less obvious examples.

\begin{theorem}[387 implies 43]\label{387_implies_43}\uses{eq387,eq43}\lean{Subgraph.Equation387_implies_Equation43}\leanok  \Equaref{387} implies \Equaref{43}.
\end{theorem}

\begin{proof}\leanok (From \href{https://mathoverflow.net/a/450905/766}{MathOverflow}).
  By \Equaref*{387}, one has the law
\begin{equation}\label{387-again}
  (x \op x) \op y = y \op x.
\end{equation}
Specializing to $y=x \op x$, we conclude
$$(x \op x) \op (x \op x) = (x \op x) \op x$$
and hence by another application of \Equaref*{387} we see that $x \op x$ is idempotent:
\begin{equation}\label{idem}
  (x \op x) \op (x \op x) = x \op x.
\end{equation}
Now, replacing $x$ by $x \op x$ in \Cref{387-again} and then using \Cref{idem} we see that
$$ (x \op x) \op y = y \op (x \op x)$$
so in particular $x \op x$ commutes with $y \op y$:
\begin{equation}\label{op-idem} (x \op x) \op (y \op y) = (y \op y) \op (x \op x).
\end{equation}
Also, from two applications of \Cref{387-again} one has
$$(x \op x) \op (y \op y) = (y \op y) \op x = x \op y.$$
Thus \Cref{op-idem} simplifies to $x \op y = y \op x$, which is \Equaref*{43}.
\end{proof}

\begin{theorem}[29 equivalent to 14]\label{29_equiv_14} \uses{eq29,eq14}\lean{Subgraph.Equation29_implies_Equation14}\leanok  \Equaref{29} is equivalent to \Equaref{14}.
\end{theorem}

This result was posed as Problem A1 from Putnam 2001.

\begin{proof}\leanok\uses{duality} By \Cref{duality} it suffices to show that \Equaref*{29} implies \Equaref*{14}.  From \Equaref*{29} one has
  $$ x = ((x \op y) \op x) \op (x \op y)$$
  and also
  $$ y = (x \op y) \op x$$
  giving $x = y \op (x \op y)$, which is \Equaref*{14}.
\end{proof}

\begin{theorem}[14 implies 29]\label{14_implies_29} \uses{eq29,eq14}\lean{Subgraph.Equation14_implies_Equation29}\leanok  \Equaref*{14} implies \Equaref*{29}.
\end{theorem}

This result was posed as Problem A1 from Putnam 2001.

\begin{proof}\leanok
\end{proof}

The following result was Problem A4 on Putnam 1978.

\begin{theorem}[3744 implies 3722, 381]\label{3744_implies_3722_381}\uses{eq3744, eq3722, eq381}\lean{Subgraph.Equation3744_implies_Equation3722, Subgraph.Equation3744_implies_Equation381}\leanok \Equaref{3744} implies \Equaref{3722} and \Equaref{381}.
\end{theorem}

\begin{proof}\leanok By hypothesis, one has
$$x \op y = (x \op z) \op (w \op y)
  $$
for all $x,y,z,w$.  Various specializations of this give
\begin{align}
 x \op y &= (x \op z) \op (y \op y) \label{381-1} \\
 x \op z &= (x \op z) \op (x \op z) \label{381-2} \\
(x \op z) \op y &= ((x \op z) \op (x \op z)) \op (y \op y) \label{381-3}.
\end{align}
\Cref{381-2} gives \Equaref*{3722}, while \Cref{381-1}, \Cref{381-2}, \Cref{381-3} gives
$$ x \op y = (x\op z) \op y$$
which is \Equaref*{381}.
\end{proof}

\begin{theorem}[1689 is equivalent to 2]\label{1689_equiv_2}\uses{eq1689, eq2}\lean{Subgraph.Equation1689_implies_Equation2, Subgraph.Equation2_implies_Equation1689}\leanok \Equaref{1689} is equivalent to \Equaref*{2}.
\end{theorem}


\begin{proof}\leanok  The implication of \Equaref*{1689} from \Equaref*{2} is trivial.  The converse is a surprisingly long chain of implications; see pages 326--327 of \cite{Kisielewicz2}.  With some computer assistance, we found the following human-readable proof.
We denote $y^1=y$ and $y^{n+1}=y^n\op y$ for $n\geq 1$.
We also introduce the notation
\begin{equation}
 f(x,y) = (x \op y) \op y , \qquad
 g(x,y) = x\op f(x,y) = x \op ((x \op y) \op y) .
\end{equation}
The initial equation states $x = (y \op x) \op f(x,z)$.
Our main step will be to prove that for all $t\in M$ there exists $w\in M$ such that $f(t,w) = t$.
The rest of the proof is then straightforward.
Indeed, the initial equation gives $t = (y \op t) \op f(t,w) = (y \op t) \op t = f(y,t)$ for any $t,y\in M$.
With such a simple expression of $f$ the initial equation becomes $x = (y \op x) \op z$, which easily implies the singleton law, for instance by writing $x = ((y \op w) \op x) \op z = w \op z$ for any $w,x,y,z \in M$.

There remains to prove $f(t,w) = t$ for a well-chosen $w \in M$, explicitly, $w=g(t,t^5)=t\op t^7$.
For any $t,u,v \in M$, the combinations $x = f(t,u)$ and $y = v \op t$ obey $y \op x = t$.  Inserting these values into the initial equation yields the identity
\begin{equation}\label{Kisielewicz-tfftuz}
f(t,u) = t \op f(f(t,u),z) .
\end{equation}
Specialize to $z=f(u,v)$ and note that $f(t,u) \op z = (\cdots \op u) \op f(u,v) = u$ by the initial equation so that $f(f(t,u),z) = (f(t,u) \op z) \op z = u \op z = g(u,v)$.  Inserting this into \eqref{Kisielewicz-tfftuz} yields
\begin{equation}\label{Kisielewicz-ftg}
f(t,u) = t \op g(u,v) .
\end{equation}

On the one hand, \eqref{Kisielewicz-tfftuz} with $z=u=t$ states that $t^3 = t \op t^5$, so (using $f(t^n,t)=t^{n+2}$)
\begin{equation}
  f(t,t^5) = (t \op t^5) \op t^5 = t^3 \op t^5 = t^3 \op f(t^3,t) = g(t^3,t) ,
\end{equation}
and \eqref{Kisielewicz-ftg} with $(u,v)=(t^3,t)$ then implies $f(t,t^3) = t \op g(t^3,t) = t \op f(t,t^5) = g(t,t^5)$.
On the other hand, \eqref{Kisielewicz-ftg} with $(u,v)=(t,t^5)$ implies $t^3 = t \op g(t,t^5)$.  We deduce
\begin{equation}
  f(t,g(t,t^5)) = (t \op g(t,t^5)) \op g(t,t^5) = t^3 \op f(t, t^3) = (\cdots \op t) \op f(t,\dots) = t .
\end{equation}
\end{proof}

The following result was established in \cite{mendelsohn-padmanabhan}.

\begin{theorem}[Consequences of 1571]\label{1571_impl}\uses{eq1571, eq2662, eq40, eq23, eq8, eq16, eq14, eq43, eq4512}\lean{Subgraph.Equation1571_implies_Equation2662, Subgraph.Equation1571_implies_Equation40, Subgraph.Equation1571_implies_Equation23,Subgraph.Equation1571_implies_Equation8, Subgraph.Equation1571_implies_Equation16, Subgraph.Equation1571_implies_Equation43, Subgraph.Equation1571_implies_Equation4512}\leanok  Magmas obeying \Equaref{1571} also obey \Equaref*{2662}, \Equaref*{40}, \Equaref*{23}, \Equaref*{8}, \Equaref*{16}, \Equaref*{14}, \Equaref*{43}, and \Equaref*{4512}, and are in fact abelian groups of exponent two.  Conversely, all abelian groups of exponent two obey \Equaref*{1571}.
\end{theorem}

\begin{proof}\leanok  Suppose that a magma $G$ obeys \Equaref*{1571}, thus
\begin{equation}\label{1571-again}
 x = (y \op z) \op (y \op (x \op z)).
\end{equation}
$$ x = ((x \op y) \op (x \op y)) \op ((x \op y) \op (x \op (x \op y)))$$
and
$$ x = (x \op y) \op (x \op (x \op y))$$
whence
$$x = ((x \op y) \op (x \op y)) \op x$$
which is \Equaref*{2662}.  This gives
$$y = ((y \op z) \op (y \op z)) \op y$$
while from \Cref{1571-again} one has
$$ (y \op z) \op (y \op z) = (x \op y) \op (x \op ((y \op z) \op (y \op z) \op y))$$
whence
$$ (x \op y) \op (x \op y) = (y \op z) \op (y \op z).$$
This implies that $(x \op y) \op (x \op y)$ does not depend on $x$, or on $y$, hence is equal to some constant $e$:
$$ (x \op y) \op (x \op y) = e.$$
From \Cref{1571-again} the magma operation is surjective, hence
\begin{equation}\label{xxe} x \op x = e
\end{equation}
which gives \Equaref*{40}.  Applying \Cref{1571-again} with $x=y=z$ we conclude
$$ x = e \op (x \op e)$$
while if we instead take $y=z=e$ we have
$$ x = e \op (e \op (x \op e))$$
hence
$$ x = e \op x$$
and then also
$$ x = x \op e$$
from which we readily conclude \Equaref*{23}, \Equaref*{8}; thus $e$ is an identity element.  From \Cref{1571-again} with $z=e$ we now have
\begin{equation}\label{16-again}
 x = y \op (y \op x)
\end{equation}
which is \Equaref*{16}. If instead we take $y=e$ we have
\begin{equation}\label{14-again}
  x = z \op (x \op z)
\end{equation}
which is \Equaref*{14}.  So if we substitute $z = x \op y$ and use \Cref{16-again} we obtain
$$ x = (x \op y) \op y$$
and hence
$$ y \op x = y \op ((x \op y) \op y) = x \op y$$
thanks to \Cref{14-again}.  This gives \Equaref*{43}, thus $G$ is now commutative.  From \Cref{1571-again} once more one has
$$x \op (y \op z) = (y \op x) \op (z \op ((x \op (y \op z)) \op x))$$
which one can simplify using commutativity and \Cref{16-again} (or \Cref{14-again}) to eventually obtain
$$x \op (y \op z) = (x \op y) \op z$$
which is \Equaref*{4512}.  $G$ is now commutative and associative, and every element is its own inverse and of exponent $2$, hence is an abelian group thanks to \Cref{xxe}, so $G$ is an abelian group of exponent $2$ as claimed.  The converse is easily verified.
\end{proof}

\begin{theorem}[953 is equivalent to 2]\label{953_equiv_2}\uses{eq953, eq2}\lean{Subgraph.Equation953_implies_Equation2}\leanok  \Equaref{953} is equivalent to \Equaref*{2}.
\end{theorem}

\begin{proof}\leanok  It suffices to show that \Equaref*{953} implies \Equaref*{2}.  Pick an element $0$ of $G$ and define $1 = 0 \op 0$ and $2 = 1 \op 1$ (we do not require $0,1,2$ to be distinct).
From \Equaref*{953} with $x=z=0$ we have
$$ 0 = y \op 2.$$
If we then apply \Equaref*{953} with $z=1$ we conclude that
$$ x = y \op 0$$
for all $x,y$, from which one concludes $x=x'$ for any $x,x' \in G$, giving \Equaref*{2}.
\end{proof}


\begin{theorem}[Sheffer stroke axiom]\label{sheffer}\uses{eq345169}\lean{Sheffer.Equation345169_is_Boolean}\leanok  Definition \Equaref{345169}
axiomatizes the Sheffer stroke operation $x \op y = \overline{xy}$ in a Boolean algebra.
\end{theorem}

\begin{proof}\leanok
See \cite{mccune_et_al}.  In fact this is the shortest law with this property.

A sketch of proof follows.  One can easily verify that the Sheffer stroke operation obeys this law.  Conversely, if this law holds, then automated theorem provers can show that the three Sheffer axioms
$$ (x \op x) \op (x \op x)  = x$$
$$ x \op (y \op (y \op y)) = x \op x$$
$$ (x \op (y \op z)) \op (x \op (y \op z)) = ((y \op y) \op x) \op ((z \op z) \op x)$$
are satisfied.  A classical result of Sheffer \cite{sheffer} then allows one to conclude.
\end{proof}

A \emph{natural central groupoid} is, up to isomorphism, a magma with carrier $S \times S$ for some set $S$ and operation
$$ (a,b) \op (c,d) = (b,c).$$
These are examples of central groupoids, \Equaref{168}.

\begin{theorem}[Natural central groupoid axiom]\label{natural-central-groupoid}\uses{eq26302} \Equaref{26302} characterizes natural central groupoids.
\end{theorem}

\begin{proof}
  See \cite[Theorem 5]{knuth}.  The proof is quite lengthy; a sketch is as follows. It is easy to see that natural central groupoids obey \Equaref*{26302}.  Conversely, if this law holds, then
\begin{align*}
  (y \op z) \op (z \op w) &= (( x \op ((w \op (y \op z)) \op w)) \op ((y \op z) \op w)) \op (z \op w)\\
  &= z
\end{align*}
so we have a central groupoid.  Setting $y = (t \op t) \op t$, $z = t \op (t \op t)$, $w = t \op t$ in \Equaref*{26302} we also obtain
$$ (x \op t) \op t = (t \op t) \op t.$$
Using the notation
$$ x^{(1)} := (x \op x) \op x, \quad x^{(2)} := x \op (x \op x)$$
we then have
\begin{align*}
  x \op t &= ((x \op x) \op (x \op t)) \op ((x \op t) \op t) \\
  &= x \op t^{(1)}.
\end{align*}
A lengthy computer-assisted argument then gave the dual identity
$$ t^{(2)} \op x = t \op x$$
Together, these give
$$ x^{(2)} \op y^{(1)} = x \op y.$$
Multiplying on the left by $x = x^{(1)}\op x^{(2)}$, one can conclude that
$$ x^{(2)} = x \op (x \op y).$$
One then has
\begin{align*}
  (x \op y)^{(1)} &= ((y \op x) \op (x \op y)) \op (x \op y) \\
&= x \op (x \op y) \\
&= x^{(2)}
\end{align*}
and a similar argument gives
$$ (x \op y)^{(2)} = y^{(1)}.$$
Since $(x \op x)^{(1)} = x^{(2)}$ and $(x \op x)^{(2)} = x^{(1)}$, we conclude that $x^{(1)}$ and $x^{(2)}$ are idempotent.  Since $x = x^{(1)} \op x^{(2)}$, we see that every $x$ is the product of two idempotents.  One can show that this representation is unique, and gives a canonical identification with a natural central groupoid.
\end{proof}
