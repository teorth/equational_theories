\chapter{Selected magmas}\label{selected-magmas-chapter}

Each magma can be used to establish anti-implications: if $\Gamma$ is the set of all laws satisfied by a magma $G$, then we have $\neg E \leq E'$ whenever $E \in \Gamma$ and $E' \not \in \Gamma$.  Large numbers of implications can already be obtained from

\begin{itemize}
  \item All magmas of size at most $4$, up to isomorphism (of which \href{https://oeis.org/A001329}{there are $178\,985\,294$});
  \item All commutative magmas of size $5$, up to isomorphism (of which \href{https://oeis.org/A001425}{there are $254\,429\,900$});
  \item Cyclic groups $\Z/N\Z$ with $2 \leq N \leq 12$ and $x \op y = ax^2+bxy+cy^2+dx+ey$ for randomly chosen $a,b,c,d,e$.
  \item There are only \href{https://oeis.org/A057991}{$1411$ quasigroups of size $5$} (up to isomorphism), and Mace4 can generate all of them in under 20 seconds. A shell script to do this is available \href{https://github.com/zaklogician/equational_theories/tree/cancellative_magmas/scripts/cancellative_magmas}{here}. A magma is a quasigroup if, for all $y$, the left/right multiplications $x\mapsto y\op x$ and $x\mapsto x\op y$ are bijective (equivalent to injective in a finite magma).
\end{itemize}

We also note that a systematic (computer-assisted) study of magmas of size $3$ was performed in \cite{berman-burris}, though with current computational resources it was feasible to iterate over all magmas of size up to $4$ by a brute force approach.

Some other magmas have been used to establish counterexamples:
\begin{itemize}
  \item The cyclic group $\Z/6\Z$ with the addition law.
  \item The natural numbers with law $x \op y = x+1$.
  \item The natural numbers with law $x \op y = xy+1$.
  \item The reals with $x \op y = (x+y)/2$.
  \item The natural numbers with $x \op y$ equal to $x$ when $x=y$ and $x+1$ otherwise.
  \item The set of strings with $x \op y$ equal to $y$ when $x=y$ or when $x$ ends with $yyy$, or $xy$ otherwise (see \href{https://leanprover.zulipchat.com/#narrow/stream/458659-Equational/topic/3102.20does.20not.20imply.203176/near/474656031}{this Zulip thread}).
  \item Vector spaces ${\mathbb F}_2^n$ over ${\mathbb F}_2$, which satisfy \Equaref{1571} (and hence all the subsequent laws mentioned in \Cref{1571_impl}).
  \item Knuth's construction \cite{knuth} of a central groupoid (\Equaref{168}) as follows.  Let $S$ be a (finite) set with a distinguished element $0$, and a binary operation $*$ such that $x*0=0$ and $0*x=x$   for all $x$, and for each $x,y$ there is a unique $z$ with $x*z=y$.  One can then define a central groupoid on $S \times S$ by defining $(a,b) \op (c,d)$ to equal $(b,c)$ if $c,d \neq 0$; $(b,e)$ if $b*e=c$ is non-zero and $d=0$; and $(a*b,0)$ if $c=0$.  One such example in \cite{knuth} is when $S = \{0,1,2\}$ with $1*1=2*1=2$ and $1*2=2*2=1$.
  \item Cancellative magmas of sizes 7 to 9, found by hand-guided search using various solvers.
  \item Two magmas of cardinality $8$ were \href{https://leanprover.zulipchat.com/#narrow/stream/458659-Equational/topic/using.20z3/near/474918100}{constructed by Z3}.
  \item A large number of ad-hoc finite magmas were constructed using the Vampire theorem prover.  In some cases, inputting theoretical information is useful: see \href{https://leanprover.zulipchat.com/#narrow/channel/458659-Equational/topic/Outstanding.20equations.2C.20v1/near/478066872}{this discussion}.
  \item Linear magmas $x\op y = ax+by$ on various fields, such as ${\mathbb F}_p$ for small primes $p$, have also been used to establish counterexamples.  One such choice is $(p,a,b) = (11,1,7)$. See \href{https://leanprover.zulipchat.com/#narrow/stream/458659-Equational/topic/An.20old.20new.20idea}{this discussion}.  For a noncommutative example, see \href{https://leanprover.zulipchat.com/#narrow/channel/458659-Equational/topic/Outstanding.20equations.2C.20v1/near/477928995.2E01}{this discussion}. For a more systematic exploration of the implications that can be obtained by both commutative and noncommutative linear models, see \href{https://leanprover.zulipchat.com/#narrow/channel/458659-Equational/topic/Non-commutative.20linear.20implications}{this discussion}.
  \item A variation of the translation-invariant magma construction which resolved the Asterix / Obelix anti-implication is used to show that \Equaref{1661} does not imply \Equaref{1657}.
\end{itemize}
