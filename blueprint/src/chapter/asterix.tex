\chapter{The Asterix equation}\label{asterix-chapter}

A \emph{translation-invariant magma} is a magma whose carrier $G$ is an abelian group $G = (G,+)$, and whose magma operation takes the form
$$ y \op x = x + f(x-y)$$
for some function $f: G \to G$.  Thus the translations on $G$ become magma isomorphisms.

For translation-invariant magmas, an equational law simplifies to a univariate functional equation.  For instance, writing $x = y+h$, we have
$$ y \op x = x + f(h)$$
$$ x \op (y \op x) = x + f(h) + f^2(h)$$
$$ y \op (x \op (y \op x)) = x + f(h) + f^2(h) + f( h + f(h) + f^2(h) )$$
where $f^2 = f \circ f$, so the Asterix equation (\Cref{eq65}) for such magmas simplifies to the univariant functional equation
\begin{equation}\label{fh}
   f(h) + f^2(h) + f( h + f(h) + f^2(h) ) = 0
\end{equation}
for $h \in G$.

This equation has some degenerate solutions, for instance we can take $f(h) = c$ for any constant $c$ of order $3$ in $G$.  It is challenging to construct more interesting solutions to this equation; however, we can do this if $G=\Z$ by a greedy algorithm.  We need the following technical definition.

\begin{definition}\label{partial-solution}  A \emph{partial solution} $(E_0, E_1, E_2, f)$ to \eqref{fh} consists of nested finite sets
$$ E_0 \subset E_1 \subset E_2 \subset \Z$$ together with a function $f: E_1 \to E_2$ with the following properties:
\begin{itemize}
  \item[(a)] If $h \in E_0$, then $f(h) \in E_1$, so that $f^2(h)$ is well-defined as an element of $E_2$; furthermore, $h + f(h) + f^2(h)$ lies in $E_1$, so that the left-hand side of \eqref{fh} makes sense; and \eqref{fh} holds.
  \item[(b)] The function $f$ is a bijection from $E_1 \backslash E_0$ to $E_2 \backslash E_1$.
\end{itemize}

We partially order the space of partial solutions to \eqref{fh} by writing $(E_0, E_1, E_2, f) \leq (E'_0, E'_1, E'_2, f')$ if the following properties hold:
\begin{itemize}
  \item $E_i \subset E'_i$ for $i=0,1,2$.
  \item $f$ agrees with $f'$ on $E_0$.
\end{itemize}
When this occurs we say that the partial solution $(E'_0, E'_1, E'_2, f')$ \emph{extends} the partial solution $(E_0, E_1, E_2, f)$.

We define the \emph{empty partial solution} $(E_0,E_1,E_2,f)$ by setting $E_0=E_1=E_2$ to be the empty set, and $f$ to be the empty function; it is the minimal element of the above partial order.
\end{definition}


We have the following iterative construction, that lets us add arbitrary elements to the core domain $E_0$:

\begin{lemma}[Enlarging a partial solution]\label{iteration}\uses{partial-solution} Let $(E_0, E_1, E_2, f)$ be a partial solution to \eqref{fh}, and let $h$ be an element of $\Z$ that does not lie in $E_0$.  Then there exists a partial solution $(E'_0, E'_1, E'_2, f')$ to \eqref{fh} that extends $(E_0, E_1, E_2, f)$, such that $h \in E_0'$.
\end{lemma}

\begin{proof}  Because $f$ maps $E_1 \backslash E_0$ bijectively to $E_2 \backslash E_1$, there are three cases:
\begin{itemize}
  \item $h$ is equal to an element $h_0$ of $G \backslash E_2$.
  \item $h$ is equal to an element $h_0$ of $E_1 \backslash E_0$.
  \item $h$ is equal to $h_1 = f(h_0)$ for some $h_0 \in E_1 \backslash E_0$, so that $h_1 \in E_2 \backslash E_1$.
\end{itemize}

We deal with these three cases in turn.

First suppose that $h = h_0\in G \backslash E_2$.  We perform the following construction.

\begin{itemize}
  \item Choose an element $h_1 \in \Z$ that does not lie in $E_2 \cup \{h_0\}$; this is possible because $E_2$ is finite.
  \item Choose an element $h_2 \in \Z$ such that $h_2, h_0+h_1+h_2$, and $-h_1-h_2$ are all distinct from each other and lie outside of $E_2 \cup \{h_0, h_1\}$; this is possible because $E_2$ is finite.
  \item Promote $h_0$ to $E_0$, promote $h_1, h_0+h_1+h_2$ to $E_1$, and promote $h_2, -h_1-h_2$ to $E_2$, creating new sets
  \begin{align*}
    E'_0 &:= E_0 \cup \{h_0\} \\
    E'_1 &:= E_1 \cup \{h_0, h_1, h_0+h_1+h_2\} \\
    E'_2 &:= E_2 \cup \{h_0, h_1, h_0+h_1+h_2, h_2, -h_1-h_2\}.
  \end{align*}
  Clearly we still have nested finite sets $E'_0 \subset E'_1 \subset E'_2$.
  \item Extend $f : E_1 \to E_0$ to a function $f': E'_1 \to E'_0$ by defining
  \begin{align*}
    f'(h_0) &:= h_1 \\
    f'(h_1) &:= h_2 \\
    f'(h_0+h_1+h_2) &:= -h_1-h_2
  \end{align*}
  while keeping $f'(h)=f(h)$ for all $h \in E_1$.
\end{itemize}
It is then a routine matter to verify that $(E'_0,E'_1,E'_2,f')$ is a partial solution to \eqref{fh} extending $(E_0,E_1,E_2,f)$ and that $E'_0$ contains $h_0$, as required.

Now suppose that $h = h_0 \in E_1 \backslash E_0$, then the quantity $h_1 := f(h_0)$ lies in $E_2 \backslash E_1$.  We perform the following variant of the above construction:
\begin{itemize}
\item Choose an element $h_2 \in \Z$ such that $h_2, h_0+h_1+h_2$, and $-h_1-h_2$ are all distinct and lie outside of $E_2$.  This is possible because $E_2$ is finite.
\item Promote $h_0$ to $E_0$, promote $h_1$ and $h_0+h_1+h_2$ to $E_1$, and promote $h_2, -h_1-h_2$ to $E_2$, thus creating new sets
\begin{align*}
  E'_0 &:= E_0 \cup \{h_0\} \\
  E'_1 &:= E_1 \cup \{h_1, h_0+h_1+h_2\} \\
  E'_2 &:= E_2 \cup \{h_0+h_1+h_2, h_2, -h_1-h_2\}.
\end{align*}
Clearly we still have nested finite sets $E'_0 \subset E'_1 \subset E'_2$.
\item Extend $f : E_1 \to E_0$ to a function $f': E'_1 \to E'_0$ by defining
\begin{align*}
  f'(h_1) &:= h_2 \\
  f'(h_0+h_1+h_2) &:= -h_1-h_2
\end{align*}
while keeping $f'(h)=f(h)$ for all $h \in E_1$.
\end{itemize}
It is then a routine matter to verify that $(E'_0,E'_1,E'_2,f')$ is a partial solution to \eqref{fh} extending $(E_0,E_1,E_2,f)$ and that $E'_0$ contains $h_0$, as required.

Finally, suppose that $h = h_1 = f(h_0)$ for some $h_0 \in E_1 \backslash E_0$, so that $h_1 \in E_2 \backslash E_1$.  Then we perform the following algorithm.
\begin{itemize}
  \item Choose an element $h_2 \in \Z$ such that $h_2, h_0+h_1+h_2$, and $-h_1-h_2$ are all distinct and lie outside of $E_2$.  This is possible because $E_2$ is finite.
  \item Choose an element $h_3 \in \Z$ such that $h_3, h_1+h_2+h_3$, and $-h_2-h_3$ are all distinct and lie outside of $E_2 \cup \{h_2, h_0+h_1+h_2,-h_1-h_2\}$.  This is possible because $E_2$ is finite.
  \item Promote $h_0, h_1$ to $E_0$, promote $h_2, h_0+h_1+h_2, h_1+h_2+h_3$ to $E_1$, and promote $h_3, -h_1-h_2,-h_2-h_3$ to $E_2$, creating new sets
\begin{align*}
  E'_0 &:= E_0 \cup \{h_0, h_1 \}\\
  E'_1 &:= E_1 \cup \{h_1, h_2, h_0+h_1+h_2, h_1+h_2+h_3 \}\\
  E'_2 &:= E_2 \cup \{h_2, h_3, h_0+h_1+h_2, h_1+h_2+h_3, -h_1-h_2, -h_2-h_3\}.
\end{align*}
Clearly we still have nested finite sets $E'_0 \subset E'_1 \subset E'_2$.
\item Extend $f : E_1 \to E_0$ to a function $f': E'_1 \to E'_0$ by defining
\begin{align*}
  f'(h_1) &:= h_2 \\
  f'(h_0+h_1+h_2) &:= -h_1-h_2
  f'(h_2) &:= h_3 \\
  f'(h_1+h_2+h_3) &:= -h_2-h_3
\end{align*}
while keeping $f'(h)=f(h)$ for all $h \in E_1$.
\end{itemize}
It is then a routine matter to verify that $(E'_0,E'_1,E'_2,f')$ is a partial solution to \eqref{fh} extending $(E_0,E_1,E_2,f)$ and that $E'_0$ contains $h_0$, as required.
\end{proof}

\begin{corollary} \label{extend}\uses{partial-solution} Every partial solution $(E_0,E_1,E_2,f)$ to \eqref{fh} can be extended to a full solution $\tilde f: \Z \to \Z$.
\end{corollary}

\begin{proof}\uses{iteration}
  If we arbitrarily well-order the integers, and iterate \Cref{iteration} to add the least element of $\Z \backslash E_0$ in this well-ordering to $E_0$, we obtain an increasing sequence $(E^{(n)}_0, E^{(n)}_1, E^{(n)}_2, f^{(n)})$ of partial solutions to \eqref{fh}, where the $E^{(n)}_0$ exhaust $\Z$: $\bigcup_{n=1}^\infty E^{(n)}_0 = \Z$.  Taking limits, we obtain a full solution $\tilde f$.
\end{proof}

\begin{corollary}\label{no-inject}  There exists a solution $f:\Z \to \Z$ to \eqref{fh} such that the map $h \mapsto h + f(h)$ is not injective.
\end{corollary}

\begin{proof}  Select integers $h_0,h_1,h_2,h'_0,h'_1,h'_2$ such that the quantities
  $$ h_0, h_1, h_2, h_0+h_1+h_2, -h_1-h_2, h'_0, h'_1, h'_2, h'_0+h'_1+h'_2, -h'_1-h'_2$$
  are all distinct, but such that
  $$ h_0 + h_1 = h'_0 + h'_1$$
  (there are many assignments of variables that accomplish this).  Then set
\begin{align*}
  E_0 &:= \{h_0, h'_0\}\\
  E_1 &:= E_0 \cup \{ h_1, h'_1, h_0+h_1+h_2, h'_0+h'_1+h'_2\}\\
  E_2 &:= E_2 \cup \{ -h_1-h_2, -h'_1-h'_2\}
\end{align*}
and define $f: E_1 \to E_2$ by the formulae
\begin{align*}
  f(h_0) &:= h_1 \\
  f(h_1) &:= h_2 \\
  f(h_0+h_1+h_2) &:= -h_1-h_2 \\
  f(h'_0) &:= h'_1 \\
  f(h'_1) &:= h'_2 \\
  f(h'_0+h'_1+h'_2) &:= -h'_1-h'_2.
\end{align*}
One can then check that $(E_0, E_1, E_2, f)$ is a partial solution to \eqref{fh}, and by construction $h \mapsto h + f(h)$ is not injective on $E_1$.  Using \Cref{iteration} to extend this partial solution to a full solution, we obtain the claim.
\end{proof}

\begin{corollary}[Asterix does not imply Obelix]\label{asterix-obelix}\uses{eq65,eq1491}  There exists a magma obeying the Asterix law (\Cref{eq65}) with carrier $\Z$ such that the left-multiplication maps $L_y: x \mapsto y \op x$ are not injective for any $y \in \Z$.  In particular, it does not obey the Obelix law (\Cref{eq1491}).
\end{corollary}

\begin{proof}\uses{no-inject} Note that $L_y (y+h) = y + h + f(h)$, so the injectivity of the left-multiplication maps is equivalent to the injectivity of the map $h \mapsto h + f(h)$.  The non-injectivity then follows from \Cref{no-inject}. Note that the Obelix law clearly expresses $x$ as a function of $y$ and $L_y x = y \op x$, forcing injectivity of left-multiplication, so the Obelix law fails.
\end{proof}

On the other hand, for finite magmas the situation is different:

\begin{proposition}[Asterix implies Obelix for finite magmas]\label{asterix-obelix-finite}\uses{eq65,eq1491}  Any finite magma obeying the Asterix law (\Cref{eq65}) also is left-cancellative and obeys the Obelix law (\Cref{eq1491}).
\end{proposition}

\begin{proof}  From \Cref{eq65} we see the map $z \mapsto y \op z$ is surjective, hence injective on a finite magma; thus the magma is left-cancellative.  Replacing $x$ by $y \op x$ in this law, we see that
$$ y \op x = y \op ((y\op x) \op (y \op (y \op x)));$$
using injectivity, we conclude
$$ x = (y\op x) \op (y \op (y \op x))$$
which is \Cref{eq1491}.
\end{proof}

We remark that a very similar argument shows that a finite magma that obeys the Obelix law has $z \mapsto y \op z$ injective, hence surjective, and then obeys the Asterix law.

\section{Obelix}
Obelix magmas are those obeying equation 1491, \Cref{eq1491}:
$$x = (y ◇ x) ◇ (y ◇ (y ◇ x))$$
Obelix is not particularly {\em structurally} similar to an Asterix, besides their shared resistance
to small counterexamples. A related set of ideas can be used to construct an Obelix that does not obey some given other equations.
The common idea is to define the magma operation
$$ x \op y = x + f(y-x) = x + f(h)$$
on some underlying Abelan group. For the case of Obelix, the resulting functional equation is
\begin{equation}\label{fh2}
  f(f^{2(h) - f(h)) = h - f(h)
\end{equation}
What differs is how we must proceed in order to satisfy this equation. We again start with a partial function $f$,
picking some initial support that will suffice to invalidate the other equation we want to show a nonimplication for.
We will progressively add more elements to the support (assuming our group is countable) until the function
is total, and we will maintain three invariants:
\begin{definition}\label{partial-solution2}  A \emph{partial solution} for an Obelix is a partial function $f : G \to G$
with the properties:
\begin{itemize}
  \item If $x$ is in the support of $f$, then so is $f(x)$.
  \item If $x$ is in the support of $f$, then so is $f^2(x) - f(x)$. This is well-defined by property (1).
  \item If $x$ in the domain of $f$, then $f(f^2(x) - f(x)) = x - f(x)$. This is well-defined by properties (1) and (2).
  \item There is an infinite number of elements {\em outside} the domain of $f$, that are all linearly independent of one another and of the support of $f$.
\end{itemize}
\end{definition}
Given a partial solution, we can try to extend it by adding an element $a$ outside of its support, and giving it a "fresh" value $f(a)$ (chosen from one of the linearly independent elements). This would invalidate the closure properties (1-3) however. In fact, as $a$, $f(a)$, $f(f(a))$, etc. as well as $f(f(a))-f(a)$ etc. must all get added at once, we need to add a countably infinite number of elements to $f$ all at once. The implications can come from closure property (1) or (2), so the implied set of values has the structure of a binary tree. This has been dubbed a "bifurcation tree".

\begin{definition}\label{bifurcation-tree} A {\em bifurcation tree} for a partial solution $f$ rooted at $a$ is an infinite binary tree. Each node is a tuple $(x,f(x))$ of new values that we will add to $f$. It is defined as follows:
\begin{itemize}
  \item The root node is $(a,x)$ for some value $x$ linearly independent of everything in the domain of $f$, and everything prior in the tree.
  \item For a node $(x,y)$, the left child is $(y,z)$, where again $y$ is linearly independent of everything so far.
  \item For a node $(x,y)$ with a left child $(y,z)$, the right child is $(z-y,x-y)$.
  \item There are infinitely many group elements linearly independent of the support of $f$ and everything in the tree.
\end{itemize}
\end{definition}
Note that when we say "prior in the tree", this is under some (any) well-ordering of the nodes in the tree, not just the ancestral pre-order: we can't be assigning the same fresh value $y$ to two cousin nodes. A bifurcation tree has the following properties, which are readily verified:

\begin{lemma}\label{valid-tree}\uses{bifurcation-tree,partial-solution2}
Suppose we have a bifurcation tree for a partial solution $f$. Then:
\begin{itemize}
  \item There is no element $(x,y)$ in the tree where $x$ is in the support of $f$.
  \item There are no two elements $(x,y)$, $(x,z)$ in the tree with the same left coordinate.
\end{itemize}
\end{lemma}
\begin{proof}
If the first property failed, it would need to be either at the root, a left child, or a right child. It cannot be the root, because the root $(a,x)$ is by assumption built from an $a$ not in the domain of $f$. It can't be a left child, because that node $(x,y)$ is the left child of some $(w,x)$ where $x$ was linearly independent of everything in $f$. And it can't be a right child, because that node $(z-y,y-x)$ was built from a $z$ that was linearly independent of $y$ and everything in $f$. So the first property must hold.

For the second property, again we can do case work on the five possibilities: root and left child, root and right child, two left children, two right children, or a left and a right child. All follow essentially the same argument, that whichever came later in the well-ordering must be linearly independent of everything available when the earlier node was constructed.
\end{proof}

This means that we can extend a partial solution by adding all of these at once:
\begin{corollary}\label{extend-with-tree}\uses{bifurcation-tree,partial-solution2,valid-tree}
Suppose we have a bifurcation tree for a partial solution $f$. Then, if all points in the tree are added as values to $f$ to build a new partial function $f'$, we have that $f'$ is also a partial solution.
\end{corollary}
\begin{proof}
This means checking that all four properties of the partial solution are still satisfied.
\begin{itemize}
  \item For any $x$ in the domain of $f'$, either it's in the domain of $f$ (and we use property (1) of $f$), or it was some node $(x,y)$ in the tree, in which case its left child $(y,z)$ gives the value for $z = f(y) = f(f(x))$.
  \item Similarly, for any $x$ in the domain of $f'$, either it's in the domain of $f$ (and we use property (2) of $f$), or is was some node $(x,y)$ in the tree, in which case its {\em right} child $(z-y,y-x)$ gives the value for $f^2(x) - f(x) = f(y) - y = z - y$.
  \item Again, for any $x$ in the domain of $f'$, either it's in the domain of $f$ (and we use property (3) of $f$), or is was some node $(x,y)$ in the tree, in which case we can look at both of its children and verify that $f(z-y) = x-y$.
  \item The last property of $f'$ is precisely what is given by the last property of bifurcation trees, that they leave infinitely many independent elements.
\end{itemize}
\end{proof}

Then we need the fact that we can always build a bifurcation tree with an arbitrary root $a$ not in the domain of $f$.
\begin{lemma}\uses{bifurcation-tree,partial-solution2,valid-tree}
  For a partial solution $f$ and an element $a$ not in its domain, there is a bifurcation tree for $f$ with root $a$.
\end{lemma}
\begin{proof}
This is just the "greedy" construction -- really, an inductive construction, where we define the $n$th node in the tree based on its parent, according to the rules given. The thing to be careful for is that we leave sufficiently many linearly independent elements at the end. If we use a standard denumeration of them, always using the first fresh element, then we will use all of them by the end, and we will fail property (4) of the bifurcation tree. One can address this by taking every {\em other} generator, or otherwise sorting the generators of the group into some easier to enumerate set. In the Lean proof, we use the Abelian group of finitely supported functions from $\mathbb{N}^2 \to \mathbb{Z}$, and then each time we build a tree we just use one "row" of the generators of the group.
\end{proof}

Finally, by repeatedly picking the least element $a$ not in the domain of $f$ and extending it with a bifurcation tree rooted at $a$, we can make an Obelix magma from any intial partial solution.
\begin{theorem}
Let $G$ be a countable Abelian group, and $f$ be a partial solution on that group. Then there is an Obelix magma $M$ that agrees with $x \op y = x + f(y-x)$ wherever the right hand side is defined.
\end{theorem}
\begin{proof}
Enumerating all the elements of $G$ will give an iterative construction for a total function $f$ that still obeys the functional equation $f(f^2(x) - f(x)) = x - f(x)$. Then this gives an Obelix magma.
\end{proof}

By picking an appropriate partial solution, we can then show that there are Obelix magmas that are not Asterix.
\begin{corollary}\uses{eq65,eq1491}
There is an Obelix magma that is not Asterix (equation 65).
\end{corollary}
\begin{proof}
This requires picking an initial set that still obeys the correct initial closure properties. TODO.
\end{proof}
